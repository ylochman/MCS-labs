
% Default to the notebook output style

    


% Inherit from the specified cell style.




    
\documentclass[11pt]{article}

    
    
    \usepackage[english,main=ukrainian]{babel}
    \usepackage{indentfirst}
    \usepackage[T1]{fontenc}
    % Nicer default font (+ math font) than Computer Modern for most use cases
    \usepackage{mathpazo}

    % Basic figure setup, for now with no caption control since it's done
    % automatically by Pandoc (which extracts ![](path) syntax from Markdown).
    \usepackage{graphicx}
    % We will generate all images so they have a width \maxwidth. This means
    % that they will get their normal width if they fit onto the page, but
    % are scaled down if they would overflow the margins.
    \makeatletter
    \def\maxwidth{\ifdim\Gin@nat@width>\linewidth\linewidth
    \else\Gin@nat@width\fi}
    \makeatother
    \let\Oldincludegraphics\includegraphics
    % Set max figure width to be 80% of text width, for now hardcoded.
    \renewcommand{\includegraphics}[1]{\Oldincludegraphics[width=.8\maxwidth]{#1}}
    % Ensure that by default, figures have no caption (until we provide a
    % proper Figure object with a Caption API and a way to capture that
    % in the conversion process - todo).
    \usepackage{caption}
    \DeclareCaptionLabelFormat{nolabel}{}
    \captionsetup{labelformat=nolabel}

    \usepackage{adjustbox} % Used to constrain images to a maximum size 
    \usepackage{xcolor} % Allow colors to be defined
    \usepackage{enumerate} % Needed for markdown enumerations to work
    \usepackage{geometry} % Used to adjust the document margins
    \usepackage{amsmath} % Equations
    \usepackage{amssymb} % Equations
    \usepackage{textcomp} % defines textquotesingle
    % Hack from http://tex.stackexchange.com/a/47451/13684:
    \AtBeginDocument{%
        \def\PYZsq{\textquotesingle}% Upright quotes in Pygmentized code
    }
    \usepackage{upquote} % Upright quotes for verbatim code
    \usepackage{eurosym} % defines \euro
    \usepackage[mathletters]{ucs} % Extended unicode (utf-8) support
    \usepackage[utf8x]{inputenc} % Allow utf-8 characters in the tex document
    \usepackage{fancyvrb} % verbatim replacement that allows latex
    \usepackage{grffile} % extends the file name processing of package graphics 
                         % to support a larger range 
    % The hyperref package gives us a pdf with properly built
    % internal navigation ('pdf bookmarks' for the table of contents,
    % internal cross-reference links, web links for URLs, etc.)
    \usepackage{hyperref}
    \usepackage{longtable} % longtable support required by pandoc >1.10
    \usepackage{booktabs}  % table support for pandoc > 1.12.2
    \usepackage[inline]{enumitem} % IRkernel/repr support (it uses the enumerate* environment)
    \usepackage[normalem]{ulem} % ulem is needed to support strikethroughs (\sout)
                                % normalem makes italics be italics, not underlines
    

    
    
    % Colors for the hyperref package
    \definecolor{urlcolor}{rgb}{0,.145,.698}
    \definecolor{linkcolor}{rgb}{.71,0.21,0.01}
    \definecolor{citecolor}{rgb}{.12,.54,.11}

    % ANSI colors
    \definecolor{ansi-black}{HTML}{3E424D}
    \definecolor{ansi-black-intense}{HTML}{282C36}
    \definecolor{ansi-red}{HTML}{E75C58}
    \definecolor{ansi-red-intense}{HTML}{B22B31}
    \definecolor{ansi-green}{HTML}{00A250}
    \definecolor{ansi-green-intense}{HTML}{007427}
    \definecolor{ansi-yellow}{HTML}{DDB62B}
    \definecolor{ansi-yellow-intense}{HTML}{B27D12}
    \definecolor{ansi-blue}{HTML}{208FFB}
    \definecolor{ansi-blue-intense}{HTML}{0065CA}
    \definecolor{ansi-magenta}{HTML}{D160C4}
    \definecolor{ansi-magenta-intense}{HTML}{A03196}
    \definecolor{ansi-cyan}{HTML}{60C6C8}
    \definecolor{ansi-cyan-intense}{HTML}{258F8F}
    \definecolor{ansi-white}{HTML}{C5C1B4}
    \definecolor{ansi-white-intense}{HTML}{A1A6B2}

    % commands and environments needed by pandoc snippets
    % extracted from the output of `pandoc -s`
    \providecommand{\tightlist}{%
      \setlength{\itemsep}{0pt}\setlength{\parskip}{0pt}}
    \DefineVerbatimEnvironment{Highlighting}{Verbatim}{commandchars=\\\{\}}
    % Add ',fontsize=\small' for more characters per line
    \newenvironment{Shaded}{}{}
    \newcommand{\KeywordTok}[1]{\textcolor[rgb]{0.00,0.44,0.13}{\textbf{{#1}}}}
    \newcommand{\DataTypeTok}[1]{\textcolor[rgb]{0.56,0.13,0.00}{{#1}}}
    \newcommand{\DecValTok}[1]{\textcolor[rgb]{0.25,0.63,0.44}{{#1}}}
    \newcommand{\BaseNTok}[1]{\textcolor[rgb]{0.25,0.63,0.44}{{#1}}}
    \newcommand{\FloatTok}[1]{\textcolor[rgb]{0.25,0.63,0.44}{{#1}}}
    \newcommand{\CharTok}[1]{\textcolor[rgb]{0.25,0.44,0.63}{{#1}}}
    \newcommand{\StringTok}[1]{\textcolor[rgb]{0.25,0.44,0.63}{{#1}}}
    \newcommand{\CommentTok}[1]{\textcolor[rgb]{0.38,0.63,0.69}{\textit{{#1}}}}
    \newcommand{\OtherTok}[1]{\textcolor[rgb]{0.00,0.44,0.13}{{#1}}}
    \newcommand{\AlertTok}[1]{\textcolor[rgb]{1.00,0.00,0.00}{\textbf{{#1}}}}
    \newcommand{\FunctionTok}[1]{\textcolor[rgb]{0.02,0.16,0.49}{{#1}}}
    \newcommand{\RegionMarkerTok}[1]{{#1}}
    \newcommand{\ErrorTok}[1]{\textcolor[rgb]{1.00,0.00,0.00}{\textbf{{#1}}}}
    \newcommand{\NormalTok}[1]{{#1}}
    
    % Additional commands for more recent versions of Pandoc
    \newcommand{\ConstantTok}[1]{\textcolor[rgb]{0.53,0.00,0.00}{{#1}}}
    \newcommand{\SpecialCharTok}[1]{\textcolor[rgb]{0.25,0.44,0.63}{{#1}}}
    \newcommand{\VerbatimStringTok}[1]{\textcolor[rgb]{0.25,0.44,0.63}{{#1}}}
    \newcommand{\SpecialStringTok}[1]{\textcolor[rgb]{0.73,0.40,0.53}{{#1}}}
    \newcommand{\ImportTok}[1]{{#1}}
    \newcommand{\DocumentationTok}[1]{\textcolor[rgb]{0.73,0.13,0.13}{\textit{{#1}}}}
    \newcommand{\AnnotationTok}[1]{\textcolor[rgb]{0.38,0.63,0.69}{\textbf{\textit{{#1}}}}}
    \newcommand{\CommentVarTok}[1]{\textcolor[rgb]{0.38,0.63,0.69}{\textbf{\textit{{#1}}}}}
    \newcommand{\VariableTok}[1]{\textcolor[rgb]{0.10,0.09,0.49}{{#1}}}
    \newcommand{\ControlFlowTok}[1]{\textcolor[rgb]{0.00,0.44,0.13}{\textbf{{#1}}}}
    \newcommand{\OperatorTok}[1]{\textcolor[rgb]{0.40,0.40,0.40}{{#1}}}
    \newcommand{\BuiltInTok}[1]{{#1}}
    \newcommand{\ExtensionTok}[1]{{#1}}
    \newcommand{\PreprocessorTok}[1]{\textcolor[rgb]{0.74,0.48,0.00}{{#1}}}
    \newcommand{\AttributeTok}[1]{\textcolor[rgb]{0.49,0.56,0.16}{{#1}}}
    \newcommand{\InformationTok}[1]{\textcolor[rgb]{0.38,0.63,0.69}{\textbf{\textit{{#1}}}}}
    \newcommand{\WarningTok}[1]{\textcolor[rgb]{0.38,0.63,0.69}{\textbf{\textit{{#1}}}}}
    
    
    % Define a nice break command that doesn't care if a line doesn't already
    % exist.
    \def\br{\hspace*{\fill} \\* }
    % Math Jax compatability definitions
    \def\gt{>}
    \def\lt{<}
    % Document parameters
    \title{Protocol}
    
    
    

    % Pygments definitions
    
\makeatletter
\def\PY@reset{\let\PY@it=\relax \let\PY@bf=\relax%
    \let\PY@ul=\relax \let\PY@tc=\relax%
    \let\PY@bc=\relax \let\PY@ff=\relax}
\def\PY@tok#1{\csname PY@tok@#1\endcsname}
\def\PY@toks#1+{\ifx\relax#1\empty\else%
    \PY@tok{#1}\expandafter\PY@toks\fi}
\def\PY@do#1{\PY@bc{\PY@tc{\PY@ul{%
    \PY@it{\PY@bf{\PY@ff{#1}}}}}}}
\def\PY#1#2{\PY@reset\PY@toks#1+\relax+\PY@do{#2}}

\expandafter\def\csname PY@tok@w\endcsname{\def\PY@tc##1{\textcolor[rgb]{0.73,0.73,0.73}{##1}}}
\expandafter\def\csname PY@tok@c\endcsname{\let\PY@it=\textit\def\PY@tc##1{\textcolor[rgb]{0.25,0.50,0.50}{##1}}}
\expandafter\def\csname PY@tok@cp\endcsname{\def\PY@tc##1{\textcolor[rgb]{0.74,0.48,0.00}{##1}}}
\expandafter\def\csname PY@tok@k\endcsname{\let\PY@bf=\textbf\def\PY@tc##1{\textcolor[rgb]{0.00,0.50,0.00}{##1}}}
\expandafter\def\csname PY@tok@kp\endcsname{\def\PY@tc##1{\textcolor[rgb]{0.00,0.50,0.00}{##1}}}
\expandafter\def\csname PY@tok@kt\endcsname{\def\PY@tc##1{\textcolor[rgb]{0.69,0.00,0.25}{##1}}}
\expandafter\def\csname PY@tok@o\endcsname{\def\PY@tc##1{\textcolor[rgb]{0.40,0.40,0.40}{##1}}}
\expandafter\def\csname PY@tok@ow\endcsname{\let\PY@bf=\textbf\def\PY@tc##1{\textcolor[rgb]{0.67,0.13,1.00}{##1}}}
\expandafter\def\csname PY@tok@nb\endcsname{\def\PY@tc##1{\textcolor[rgb]{0.00,0.50,0.00}{##1}}}
\expandafter\def\csname PY@tok@nf\endcsname{\def\PY@tc##1{\textcolor[rgb]{0.00,0.00,1.00}{##1}}}
\expandafter\def\csname PY@tok@nc\endcsname{\let\PY@bf=\textbf\def\PY@tc##1{\textcolor[rgb]{0.00,0.00,1.00}{##1}}}
\expandafter\def\csname PY@tok@nn\endcsname{\let\PY@bf=\textbf\def\PY@tc##1{\textcolor[rgb]{0.00,0.00,1.00}{##1}}}
\expandafter\def\csname PY@tok@ne\endcsname{\let\PY@bf=\textbf\def\PY@tc##1{\textcolor[rgb]{0.82,0.25,0.23}{##1}}}
\expandafter\def\csname PY@tok@nv\endcsname{\def\PY@tc##1{\textcolor[rgb]{0.10,0.09,0.49}{##1}}}
\expandafter\def\csname PY@tok@no\endcsname{\def\PY@tc##1{\textcolor[rgb]{0.53,0.00,0.00}{##1}}}
\expandafter\def\csname PY@tok@nl\endcsname{\def\PY@tc##1{\textcolor[rgb]{0.63,0.63,0.00}{##1}}}
\expandafter\def\csname PY@tok@ni\endcsname{\let\PY@bf=\textbf\def\PY@tc##1{\textcolor[rgb]{0.60,0.60,0.60}{##1}}}
\expandafter\def\csname PY@tok@na\endcsname{\def\PY@tc##1{\textcolor[rgb]{0.49,0.56,0.16}{##1}}}
\expandafter\def\csname PY@tok@nt\endcsname{\let\PY@bf=\textbf\def\PY@tc##1{\textcolor[rgb]{0.00,0.50,0.00}{##1}}}
\expandafter\def\csname PY@tok@nd\endcsname{\def\PY@tc##1{\textcolor[rgb]{0.67,0.13,1.00}{##1}}}
\expandafter\def\csname PY@tok@s\endcsname{\def\PY@tc##1{\textcolor[rgb]{0.73,0.13,0.13}{##1}}}
\expandafter\def\csname PY@tok@sd\endcsname{\let\PY@it=\textit\def\PY@tc##1{\textcolor[rgb]{0.73,0.13,0.13}{##1}}}
\expandafter\def\csname PY@tok@si\endcsname{\let\PY@bf=\textbf\def\PY@tc##1{\textcolor[rgb]{0.73,0.40,0.53}{##1}}}
\expandafter\def\csname PY@tok@se\endcsname{\let\PY@bf=\textbf\def\PY@tc##1{\textcolor[rgb]{0.73,0.40,0.13}{##1}}}
\expandafter\def\csname PY@tok@sr\endcsname{\def\PY@tc##1{\textcolor[rgb]{0.73,0.40,0.53}{##1}}}
\expandafter\def\csname PY@tok@ss\endcsname{\def\PY@tc##1{\textcolor[rgb]{0.10,0.09,0.49}{##1}}}
\expandafter\def\csname PY@tok@sx\endcsname{\def\PY@tc##1{\textcolor[rgb]{0.00,0.50,0.00}{##1}}}
\expandafter\def\csname PY@tok@m\endcsname{\def\PY@tc##1{\textcolor[rgb]{0.40,0.40,0.40}{##1}}}
\expandafter\def\csname PY@tok@gh\endcsname{\let\PY@bf=\textbf\def\PY@tc##1{\textcolor[rgb]{0.00,0.00,0.50}{##1}}}
\expandafter\def\csname PY@tok@gu\endcsname{\let\PY@bf=\textbf\def\PY@tc##1{\textcolor[rgb]{0.50,0.00,0.50}{##1}}}
\expandafter\def\csname PY@tok@gd\endcsname{\def\PY@tc##1{\textcolor[rgb]{0.63,0.00,0.00}{##1}}}
\expandafter\def\csname PY@tok@gi\endcsname{\def\PY@tc##1{\textcolor[rgb]{0.00,0.63,0.00}{##1}}}
\expandafter\def\csname PY@tok@gr\endcsname{\def\PY@tc##1{\textcolor[rgb]{1.00,0.00,0.00}{##1}}}
\expandafter\def\csname PY@tok@ge\endcsname{\let\PY@it=\textit}
\expandafter\def\csname PY@tok@gs\endcsname{\let\PY@bf=\textbf}
\expandafter\def\csname PY@tok@gp\endcsname{\let\PY@bf=\textbf\def\PY@tc##1{\textcolor[rgb]{0.00,0.00,0.50}{##1}}}
\expandafter\def\csname PY@tok@go\endcsname{\def\PY@tc##1{\textcolor[rgb]{0.53,0.53,0.53}{##1}}}
\expandafter\def\csname PY@tok@gt\endcsname{\def\PY@tc##1{\textcolor[rgb]{0.00,0.27,0.87}{##1}}}
\expandafter\def\csname PY@tok@err\endcsname{\def\PY@bc##1{\setlength{\fboxsep}{0pt}\fcolorbox[rgb]{1.00,0.00,0.00}{1,1,1}{\strut ##1}}}
\expandafter\def\csname PY@tok@kc\endcsname{\let\PY@bf=\textbf\def\PY@tc##1{\textcolor[rgb]{0.00,0.50,0.00}{##1}}}
\expandafter\def\csname PY@tok@kd\endcsname{\let\PY@bf=\textbf\def\PY@tc##1{\textcolor[rgb]{0.00,0.50,0.00}{##1}}}
\expandafter\def\csname PY@tok@kn\endcsname{\let\PY@bf=\textbf\def\PY@tc##1{\textcolor[rgb]{0.00,0.50,0.00}{##1}}}
\expandafter\def\csname PY@tok@kr\endcsname{\let\PY@bf=\textbf\def\PY@tc##1{\textcolor[rgb]{0.00,0.50,0.00}{##1}}}
\expandafter\def\csname PY@tok@bp\endcsname{\def\PY@tc##1{\textcolor[rgb]{0.00,0.50,0.00}{##1}}}
\expandafter\def\csname PY@tok@fm\endcsname{\def\PY@tc##1{\textcolor[rgb]{0.00,0.00,1.00}{##1}}}
\expandafter\def\csname PY@tok@vc\endcsname{\def\PY@tc##1{\textcolor[rgb]{0.10,0.09,0.49}{##1}}}
\expandafter\def\csname PY@tok@vg\endcsname{\def\PY@tc##1{\textcolor[rgb]{0.10,0.09,0.49}{##1}}}
\expandafter\def\csname PY@tok@vi\endcsname{\def\PY@tc##1{\textcolor[rgb]{0.10,0.09,0.49}{##1}}}
\expandafter\def\csname PY@tok@vm\endcsname{\def\PY@tc##1{\textcolor[rgb]{0.10,0.09,0.49}{##1}}}
\expandafter\def\csname PY@tok@sa\endcsname{\def\PY@tc##1{\textcolor[rgb]{0.73,0.13,0.13}{##1}}}
\expandafter\def\csname PY@tok@sb\endcsname{\def\PY@tc##1{\textcolor[rgb]{0.73,0.13,0.13}{##1}}}
\expandafter\def\csname PY@tok@sc\endcsname{\def\PY@tc##1{\textcolor[rgb]{0.73,0.13,0.13}{##1}}}
\expandafter\def\csname PY@tok@dl\endcsname{\def\PY@tc##1{\textcolor[rgb]{0.73,0.13,0.13}{##1}}}
\expandafter\def\csname PY@tok@s2\endcsname{\def\PY@tc##1{\textcolor[rgb]{0.73,0.13,0.13}{##1}}}
\expandafter\def\csname PY@tok@sh\endcsname{\def\PY@tc##1{\textcolor[rgb]{0.73,0.13,0.13}{##1}}}
\expandafter\def\csname PY@tok@s1\endcsname{\def\PY@tc##1{\textcolor[rgb]{0.73,0.13,0.13}{##1}}}
\expandafter\def\csname PY@tok@mb\endcsname{\def\PY@tc##1{\textcolor[rgb]{0.40,0.40,0.40}{##1}}}
\expandafter\def\csname PY@tok@mf\endcsname{\def\PY@tc##1{\textcolor[rgb]{0.40,0.40,0.40}{##1}}}
\expandafter\def\csname PY@tok@mh\endcsname{\def\PY@tc##1{\textcolor[rgb]{0.40,0.40,0.40}{##1}}}
\expandafter\def\csname PY@tok@mi\endcsname{\def\PY@tc##1{\textcolor[rgb]{0.40,0.40,0.40}{##1}}}
\expandafter\def\csname PY@tok@il\endcsname{\def\PY@tc##1{\textcolor[rgb]{0.40,0.40,0.40}{##1}}}
\expandafter\def\csname PY@tok@mo\endcsname{\def\PY@tc##1{\textcolor[rgb]{0.40,0.40,0.40}{##1}}}
\expandafter\def\csname PY@tok@ch\endcsname{\let\PY@it=\textit\def\PY@tc##1{\textcolor[rgb]{0.25,0.50,0.50}{##1}}}
\expandafter\def\csname PY@tok@cm\endcsname{\let\PY@it=\textit\def\PY@tc##1{\textcolor[rgb]{0.25,0.50,0.50}{##1}}}
\expandafter\def\csname PY@tok@cpf\endcsname{\let\PY@it=\textit\def\PY@tc##1{\textcolor[rgb]{0.25,0.50,0.50}{##1}}}
\expandafter\def\csname PY@tok@c1\endcsname{\let\PY@it=\textit\def\PY@tc##1{\textcolor[rgb]{0.25,0.50,0.50}{##1}}}
\expandafter\def\csname PY@tok@cs\endcsname{\let\PY@it=\textit\def\PY@tc##1{\textcolor[rgb]{0.25,0.50,0.50}{##1}}}

\def\PYZbs{\char`\\}
\def\PYZus{\char`\_}
\def\PYZob{\char`\{}
\def\PYZcb{\char`\}}
\def\PYZca{\char`\^}
\def\PYZam{\char`\&}
\def\PYZlt{\char`\<}
\def\PYZgt{\char`\>}
\def\PYZsh{\char`\#}
\def\PYZpc{\char`\%}
\def\PYZdl{\char`\$}
\def\PYZhy{\char`\-}
\def\PYZsq{\char`\'}
\def\PYZdq{\char`\"}
\def\PYZti{\char`\~}
% for compatibility with earlier versions
\def\PYZat{@}
\def\PYZlb{[}
\def\PYZrb{]}
\makeatother


    % Exact colors from NB
    \definecolor{incolor}{rgb}{0.0, 0.0, 0.5}
    \definecolor{outcolor}{rgb}{0.545, 0.0, 0.0}



    
    % Prevent overflowing lines due to hard-to-break entities
    \sloppy 
    % Setup hyperref package
    \hypersetup{
      breaklinks=true,  % so long urls are correctly broken across lines
      colorlinks=true,
      urlcolor=urlcolor,
      linkcolor=linkcolor,
      citecolor=citecolor,
      }
    % Slightly bigger margins than the latex defaults
    
    \geometry{verbose,tmargin=1in,bmargin=1in,lmargin=1in,rmargin=1in}
    
    

    \begin{document}
    
\begin{titlepage} 
\center

{\large
Національний Технічний Університет України\\[0.2cm]
«Київський політехнічний інститут імені Ігоря Сікорського»\\[0.2cm]
Навчально-Науковий Комплекс \\[0.2cm]
«Інститут прикладного системного аналізу»
}\\[4cm] % Minor heading such as course title


{ \LARGE \bfseries
Лабороторна робота № 1 \\[0.4cm]
з дисципліни «Моделювання складних систем»\\[0.4cm]
Тема: «Динаміка процесів, що описуються диференціальними рівняннями першого та другого порядку»
}\\[5cm] % Title of your document

\begin{minipage}{0.4\textwidth}
\begin{flushleft} \large
\emph{Виконали:} \\
студенти гр. КА-41 \\
Мельничук Валентин \\
Лочман Ярослава  \\
Снігірьова Валерія
\end{flushleft}
\end{minipage}
~
\begin{minipage}{0.4\textwidth}
\begin{flushright} \large
\emph{Прийняв:} \\
професор кафедри ММСА, \\
д.т.н. Степашко В.С. \\
\end{flushright}
\end{minipage}\\[5cm]

Київ 2018

\vfill % Fill the rest of the page with whitespace

\end{titlepage}

    
    

    
    \section{Опис та аналіз математичних
моделей}\label{ux43eux43fux438ux441-ux442ux430-ux430ux43dux430ux43bux456ux437-ux43cux430ux442ux435ux43cux430ux442ux438ux447ux43dux438ux445-ux43cux43eux434ux435ux43bux435ux439}

    \subsection{Динаміка росту популяції (модель
Фергюльста)}\label{ux434ux438ux43dux430ux43cux456ux43aux430-ux440ux43eux441ux442ux443-ux43fux43eux43fux443ux43bux44fux446ux456ux457-ux43cux43eux434ux435ux43bux44c-ux444ux435ux440ux433ux44eux43bux44cux441ux442ux430}

    Описує динаміку росту популяції з урахуванням обмеженості ресурсів
середовища:

$ N'=μN(k-N) \enspace \textbar{} \enspace N_0 $

$ N $ -- чисельність популяції

$ µ $ -- коефіцієнти народжуваності (приріст популяції за одиницю
часу)

$ k $ -- ємність середовища (гранична чисельність популяції)

$ N_0 $ -- початкова чисельність популяції

    Стаціонарні точки:

$ N'= 0 \enspace \Leftrightarrow \enspace μN(k-N) = 0 $

\[ \Rightarrow \left \{\begin{array}{ll}
                        N_{1,стац} = 0 \\
                        N_{2,стац} = k
                       \end{array}
                       \right. \]

    Точка перегину:

$ N'' = μN'(k-N)-μNN' = μN'((k-2N) = μ^2N(k-N)(k-2N) $

$ N'' = 0 \enspace \Leftrightarrow \enspace μ^2N(k-N)(k-2N) = 0 $

\[ \Rightarrow \left \{\begin{array}{ll}
                        N_{1} = 0 \\
                        N_{2} = k/2 \\
                        N_{3} = k \\
                        \text{немає сенсу розглядати } N<0 \\
                        N = k \text{ - стаціонарна точка}
                       \end{array}
                       \right. \]

$ \Rightarrow N_\text{перегин} = k/2 $

    \subsubsection{Зростаючий
процес}\label{ux437ux440ux43eux441ux442ux430ux44eux447ux438ux439-ux43fux440ux43eux446ux435ux441}

    \begin{Verbatim}[commandchars=\\\{\}]
{\color{incolor}In [{\color{incolor}62}]:} \PY{n}{model1}\PY{p}{(}\PY{l+m+mi}{15}\PY{p}{,} \PY{l+m+mf}{0.05}\PY{p}{,} \PY{l+m+mi}{10}\PY{p}{)}
\end{Verbatim}


    \begin{Verbatim}[commandchars=\\\{\}]
Ємність середовища (гранична чисельність): k = 15
Швидкість росту популяції: μ = 0.05
Початковий розмір популяції: N0 = 10

    \end{Verbatim}

    \begin{center}
    \adjustimage{max size={0.9\linewidth}{0.9\paperheight}}{output_6_1.png}
    \end{center}
    { \hspace*{\fill} \\}
    
    При \(N0 < k\) та \(\mu > 0\) відбуватиметься зростання популяції аж до
граничної чисельності.

    \begin{Verbatim}[commandchars=\\\{\}]
{\color{incolor}In [{\color{incolor}64}]:} \PY{n}{model1}\PY{p}{(}\PY{l+m+mi}{15}\PY{p}{,} \PY{l+m+mf}{0.05}\PY{p}{,} \PY{l+m+mf}{0.5}\PY{p}{)}
\end{Verbatim}


    \begin{Verbatim}[commandchars=\\\{\}]
Ємність середовища (гранична чисельність): k = 15
Швидкість росту популяції: μ = 0.05
Початковий розмір популяції: N0 = 0.5

    \end{Verbatim}

    \begin{center}
    \adjustimage{max size={0.9\linewidth}{0.9\paperheight}}{output_8_1.png}
    \end{center}
    { \hspace*{\fill} \\}
    
    При цьому до значення чисельності \(k/2\) швидкість зростання
збільшується, а після нього зменшується до нуля (до чисельності
популяції \(N=k\)) - дійсно, вище це було виведено аналітично, і
підтверджено експериментально.

    \subsubsection{Стаціонарний
процес}\label{ux441ux442ux430ux446ux456ux43eux43dux430ux440ux43dux438ux439-ux43fux440ux43eux446ux435ux441}

    \begin{Verbatim}[commandchars=\\\{\}]
{\color{incolor}In [{\color{incolor}66}]:} \PY{n}{model1}\PY{p}{(}\PY{l+m+mi}{100}\PY{p}{,} \PY{l+m+mf}{0.05}\PY{p}{,} \PY{l+m+mi}{0}\PY{p}{)}
\end{Verbatim}


    \begin{Verbatim}[commandchars=\\\{\}]
Ємність середовища (гранична чисельність): k = 100
Швидкість росту популяції: μ = 0.05
Початковий розмір популяції: N0 = 0

    \end{Verbatim}

    \begin{center}
    \adjustimage{max size={0.9\linewidth}{0.9\paperheight}}{output_11_1.png}
    \end{center}
    { \hspace*{\fill} \\}
    
    При початковій нульовій чисельності популяції з часом нічого не
змінюється. Дійсно, розмноження популяції не відбувається через її
відсутність.

    \begin{Verbatim}[commandchars=\\\{\}]
{\color{incolor}In [{\color{incolor}67}]:} \PY{n}{model1}\PY{p}{(}\PY{l+m+mi}{100}\PY{p}{,} \PY{l+m+mf}{0.05}\PY{p}{,} \PY{l+m+mi}{100}\PY{p}{)}
\end{Verbatim}


    \begin{Verbatim}[commandchars=\\\{\}]
Ємність середовища (гранична чисельність): k = 100
Швидкість росту популяції: μ = 0.05
Початковий розмір популяції: N0 = 100

    \end{Verbatim}

    \begin{center}
    \adjustimage{max size={0.9\linewidth}{0.9\paperheight}}{output_13_1.png}
    \end{center}
    { \hspace*{\fill} \\}
    
    За початкової чисельності, рівної граничній чисельності, через
обмеженість ресурсів популяція далі зростати не може, тож вона
залишається такою й надалі.

    \subsubsection{Спадний
процес}\label{ux441ux43fux430ux434ux43dux438ux439-ux43fux440ux43eux446ux435ux441}

    \begin{Verbatim}[commandchars=\\\{\}]
{\color{incolor}In [{\color{incolor}68}]:} \PY{n}{model1}\PY{p}{(}\PY{l+m+mi}{100}\PY{p}{,} \PY{l+m+mf}{0.001}\PY{p}{,} \PY{l+m+mi}{150}\PY{p}{)}
\end{Verbatim}


    \begin{Verbatim}[commandchars=\\\{\}]
Ємність середовища (гранична чисельність): k = 100
Швидкість росту популяції: μ = 0.001
Початковий розмір популяції: N0 = 150

    \end{Verbatim}

    \begin{center}
    \adjustimage{max size={0.9\linewidth}{0.9\paperheight}}{output_16_1.png}
    \end{center}
    { \hspace*{\fill} \\}
    
    Коли \(N_0 > k\), то спостерігається процес спадання \(N\) до значення
\(k\). Це так само пов'язано з обмеженістю ресурсів.

    \subsubsection{Повне
вимирання}\label{ux43fux43eux432ux43dux435-ux432ux438ux43cux438ux440ux430ux43dux43dux44f}

    \begin{Verbatim}[commandchars=\\\{\}]
{\color{incolor}In [{\color{incolor}69}]:} \PY{n}{model1}\PY{p}{(}\PY{l+m+mi}{100}\PY{p}{,} \PY{o}{\PYZhy{}}\PY{l+m+mf}{0.001}\PY{p}{,} \PY{l+m+mi}{10}\PY{p}{)}
\end{Verbatim}


    \begin{Verbatim}[commandchars=\\\{\}]
Ємність середовища (гранична чисельність): k = 100
Швидкість росту популяції: μ = -0.001
Початковий розмір популяції: N0 = 10

    \end{Verbatim}

    \begin{center}
    \adjustimage{max size={0.9\linewidth}{0.9\paperheight}}{output_19_1.png}
    \end{center}
    { \hspace*{\fill} \\}
    
    Якщо швидкість росту популяції від'ємна (\(\mu < 0\)), тобто коли
природна смертність більша за природну народжуваність, то при будь-якому
\(N_0 < k\) буде спостерігатися вимирання популяції (спадання \(N\) до
нуля).

    \subsection{Односекторна модель економічної динаміки (модель
Солоу)}\label{ux43eux434ux43dux43eux441ux435ux43aux442ux43eux440ux43dux430-ux43cux43eux434ux435ux43bux44c-ux435ux43aux43eux43dux43eux43cux456ux447ux43dux43eux457-ux434ux438ux43dux430ux43cux456ux43aux438-ux43cux43eux434ux435ux43bux44c-ux441ux43eux43bux43eux443}

    Економічна система виробляє один продукт, який як споживається, так і
інвестується. Експорт і імпорт не враховуються. Стан економіки в моделі
Солоу визначається за допомогою п'яти ендогенних змінних, що змінюються
з часом

$ k'=sak^\alpha-(\mu+g)k $

$ a $ -- рівень розвитку економіки

$ 0 \textless{} \alpha \textless{} 1 $ -- частка капіталу в продукції

$ 0 \textless{} \mu \textless{} 1 $ -- норма витрат на амортизацію

$ 0 \textless{} s \textless{} 1 $ -- норма накопичення

$ 0 \textless{} g \textless{} 1 $ -- темп приросту робочої сили

$ k_0 $ -- початковий капітал

    Стаціонарний розв'язок моделі економічної динаміки залежить від
параметрів рівняння, початкових умов, а також конкретного варіанта
виробничої функції. В цій роботі ми використовуємо виробничу функцію
Кобба-Дугласа. Знайдемо для цього випадку стаціонарний розв'язок:

$ k'= 0 \enspace \Leftrightarrow \enspace sak^\alpha-(\mu+g)k = 0 $

Позначимо \(\frac{sa}{\mu+g} = c\)

$ \Rightarrow ck^\alpha-k = 0 $

Обидві функції ( $ ck^\alpha $ і $ k $ ) є монотонними функціями,
і перша - увігнута, а друга - опулка (і одночасно увігнута, до речі).
Вони перетинаються в нулі та ще одній точці:

$ ck^{\alpha-1}-1 = 0
\enspace \Rightarrow \enspace k^{1-\alpha} = c
\enspace \Rightarrow \enspace k = c^{\frac{1}{1-\alpha}} $

\[ \Rightarrow \left \{\begin{array}{ll}
                        k_{1,стац} = 0 \\
                        k_{2,стац} = (\frac{sa}{\mu+g})^{\frac{1}{1-\alpha}}
                       \end{array}
                       \right. \]

    \subsubsection{Значення вхідних параметрів для
України}\label{ux437ux43dux430ux447ux435ux43dux43dux44f-ux432ux445ux456ux434ux43dux438ux445-ux43fux430ux440ux430ux43cux435ux442ux440ux456ux432-ux434ux43bux44f-ux443ux43aux440ux430ux457ux43dux438}

    \begin{Verbatim}[commandchars=\\\{\}]
{\color{incolor}In [{\color{incolor}71}]:} \PY{n}{model2}\PY{p}{(}\PY{l+m+mf}{0.2}\PY{p}{,} \PY{l+m+mf}{2.5}\PY{p}{,} \PY{l+m+mf}{0.3}\PY{p}{,} \PY{l+m+mf}{0.1}\PY{p}{,} \PY{l+m+mf}{0.1}\PY{p}{,} \PY{l+m+mf}{0.8}\PY{p}{)}
\end{Verbatim}


    \begin{Verbatim}[commandchars=\\\{\}]
Норма накопичення: s = 0.2
Рівень розвитку економіки: a = 2.5
Частка капіталу у сукупній продукції: α = 0.3
Норма витрат на амортизацію: μ = 0.1
Темп приросту робочої сили: q = 0.1
Початкова капіталозабезпеченість: k0 = 0.8

    \end{Verbatim}

    \begin{center}
    \adjustimage{max size={0.9\linewidth}{0.9\paperheight}}{output_25_1.png}
    \end{center}
    { \hspace*{\fill} \\}
    
    Процес не містить точки перегину - капіталоозброєність буде зростати,
але швидкість зростання зменшуватиметься.

    \subsubsection{Зростаючий
процес}\label{ux437ux440ux43eux441ux442ux430ux44eux447ux438ux439-ux43fux440ux43eux446ux435ux441}

    \begin{Verbatim}[commandchars=\\\{\}]
{\color{incolor}In [{\color{incolor}72}]:} \PY{n}{model2}\PY{p}{(}\PY{l+m+mf}{0.2}\PY{p}{,} \PY{l+m+mf}{2.5}\PY{p}{,} \PY{l+m+mf}{0.8}\PY{p}{,} \PY{l+m+mf}{0.1}\PY{p}{,} \PY{l+m+mf}{0.1}\PY{p}{,} \PY{l+m+mf}{0.1}\PY{p}{)}
\end{Verbatim}


    \begin{Verbatim}[commandchars=\\\{\}]
Норма накопичення: s = 0.2
Рівень розвитку економіки: a = 2.5
Частка капіталу у сукупній продукції: α = 0.8
Норма витрат на амортизацію: μ = 0.1
Темп приросту робочої сили: q = 0.1
Початкова капіталозабезпеченість: k0 = 0.1

    \end{Verbatim}

    \begin{center}
    \adjustimage{max size={0.9\linewidth}{0.9\paperheight}}{output_28_1.png}
    \end{center}
    { \hspace*{\fill} \\}
    
    Приклад зростаючого розв'язку, де наявна точка перегину, до якої
відбувається пришвидшене зростання. На відміну від параметрів для
України, в цьому процесі було збільшено лише частку капіталу у сукіпній
продукції, що дало суттєві результати (капіталоозброєність при \(t=100\)
зросла майже втричі).

    \begin{Verbatim}[commandchars=\\\{\}]
{\color{incolor}In [{\color{incolor}73}]:} \PY{n}{model2}\PY{p}{(}\PY{l+m+mf}{0.5}\PY{p}{,} \PY{l+m+mf}{2.7}\PY{p}{,} \PY{l+m+mf}{0.9}\PY{p}{,} \PY{l+m+mf}{0.1}\PY{p}{,} \PY{l+m+mf}{0.1}\PY{p}{,} \PY{l+m+mf}{0.8}\PY{p}{)}
\end{Verbatim}


    \begin{Verbatim}[commandchars=\\\{\}]
Норма накопичення: s = 0.5
Рівень розвитку економіки: a = 2.7
Частка капіталу у сукупній продукції: α = 0.9
Норма витрат на амортизацію: μ = 0.1
Темп приросту робочої сили: q = 0.1
Початкова капіталозабезпеченість: k0 = 0.8

    \end{Verbatim}

    \begin{center}
    \adjustimage{max size={0.9\linewidth}{0.9\paperheight}}{output_30_1.png}
    \end{center}
    { \hspace*{\fill} \\}
    
    При збільшенні норми накопичення, рівня розвитку економіки або частки
капіталу у сукупній продукції можна суттєво збільшити
капіталоозброєність та швидкість її зростання.

    \subsubsection{Стаціонарний
процес}\label{ux441ux442ux430ux446ux456ux43eux43dux430ux440ux43dux438ux439-ux43fux440ux43eux446ux435ux441}

    \begin{Verbatim}[commandchars=\\\{\}]
{\color{incolor}In [{\color{incolor}74}]:} \PY{n}{model2}\PY{p}{(}\PY{l+m+mf}{0.2}\PY{p}{,} \PY{l+m+mf}{2.5}\PY{p}{,} \PY{l+m+mf}{0.3}\PY{p}{,} \PY{l+m+mf}{0.1}\PY{p}{,} \PY{l+m+mf}{0.1}\PY{p}{,} \PY{l+m+mi}{0}\PY{p}{)}
\end{Verbatim}


    \begin{Verbatim}[commandchars=\\\{\}]
Норма накопичення: s = 0.2
Рівень розвитку економіки: a = 2.5
Частка капіталу у сукупній продукції: α = 0.3
Норма витрат на амортизацію: μ = 0.1
Темп приросту робочої сили: q = 0.1
Початкова капіталозабезпеченість: k0 = 0

    \end{Verbatim}

    \begin{center}
    \adjustimage{max size={0.9\linewidth}{0.9\paperheight}}{output_33_1.png}
    \end{center}
    { \hspace*{\fill} \\}
    
    Тривіальний стаціонарний розв'язок - \(k_0 = 0\) (відсутність
інвестицій)

    \begin{Verbatim}[commandchars=\\\{\}]
{\color{incolor}In [{\color{incolor}75}]:} \PY{n}{s} \PY{o}{=} \PY{l+m+mf}{0.2}
         \PY{n}{a} \PY{o}{=} \PY{l+m+mf}{2.5}
         \PY{n}{alpha} \PY{o}{=} \PY{l+m+mf}{0.3}
         \PY{n}{mu} \PY{o}{=} \PY{l+m+mf}{0.1}
         \PY{n}{g} \PY{o}{=} \PY{l+m+mf}{0.1}
         \PY{n}{k} \PY{o}{=} \PY{p}{(}\PY{n}{s} \PY{o}{*} \PY{n}{a} \PY{o}{/} \PY{p}{(}\PY{n}{mu} \PY{o}{+} \PY{n}{g}\PY{p}{)}\PY{p}{)} \PY{o}{*}\PY{o}{*} \PY{p}{(}\PY{l+m+mi}{1} \PY{o}{/} \PY{p}{(}\PY{l+m+mi}{1} \PY{o}{\PYZhy{}} \PY{n}{alpha}\PY{p}{)}\PY{p}{)}
         \PY{n}{model2}\PY{p}{(}\PY{n}{s}\PY{p}{,} \PY{n}{a}\PY{p}{,} \PY{n}{alpha}\PY{p}{,} \PY{n}{mu}\PY{p}{,} \PY{n}{g}\PY{p}{,} \PY{n}{k}\PY{p}{)}
\end{Verbatim}


    \begin{Verbatim}[commandchars=\\\{\}]
Норма накопичення: s = 0.2
Рівень розвитку економіки: a = 2.5
Частка капіталу у сукупній продукції: α = 0.3
Норма витрат на амортизацію: μ = 0.1
Темп приросту робочої сили: q = 0.1
Початкова капіталозабезпеченість: k0 = 3.7024203699314673

    \end{Verbatim}

    \begin{center}
    \adjustimage{max size={0.9\linewidth}{0.9\paperheight}}{output_35_1.png}
    \end{center}
    { \hspace*{\fill} \\}
    
    Стаціонарний розв'язок при
\(k_0 = (\frac{sa}{\mu+g})^{\frac{1}{1-\alpha}}\) - капіталоозброєність
залишатиметься такою і надалі.

    \subsubsection{Спадний
процес}\label{ux441ux43fux430ux434ux43dux438ux439-ux43fux440ux43eux446ux435ux441}

    \begin{Verbatim}[commandchars=\\\{\}]
{\color{incolor}In [{\color{incolor}76}]:} \PY{n}{model2}\PY{p}{(}\PY{l+m+mf}{0.2}\PY{p}{,} \PY{l+m+mf}{2.5}\PY{p}{,} \PY{l+m+mf}{0.3}\PY{p}{,} \PY{l+m+mf}{0.1}\PY{p}{,} \PY{l+m+mf}{0.1}\PY{p}{,} \PY{l+m+mf}{3.8}\PY{p}{)}
\end{Verbatim}


    \begin{Verbatim}[commandchars=\\\{\}]
Норма накопичення: s = 0.2
Рівень розвитку економіки: a = 2.5
Частка капіталу у сукупній продукції: α = 0.3
Норма витрат на амортизацію: μ = 0.1
Темп приросту робочої сили: q = 0.1
Початкова капіталозабезпеченість: k0 = 3.8

    \end{Verbatim}

    \begin{center}
    \adjustimage{max size={0.9\linewidth}{0.9\paperheight}}{output_38_1.png}
    \end{center}
    { \hspace*{\fill} \\}
    
    Коли початкова капіталозброєність більша, ніж стаціонарний стан
(\(k_0 > (\frac{sa}{\mu+g})^{\frac{1}{1-\alpha}}\)), спостерігається
явище "проїдання фондів", коли при недостатньо високому рівні економіки
і невеликій частці капіталу у сукупній продукції вкладають багато
інвестицій.

    \subsection{Вимушені
коливання}\label{ux432ux438ux43cux443ux448ux435ux43dux456-ux43aux43eux43bux438ux432ux430ux43dux43dux44f}

    $ x''+2\delta x'+\omega_0^2 x=0 $

$ \delta $ - коефіцієнт згасання

$ ω_0 $ - власна частота коливань

$ ω $ -- частота коливань зовнішньої сили

$ f_0 $ -- амплітуда зовнішньої сили

\[ \left \{\begin{array}{ll}
           \ddot{x}(t)+2\delta \dot{x}(t)+\omega_0^2 x(t)=0  \\
           x(t) = A cos(\omega t+\phi); \enspace \dot{x}(t) = - A \omega sin(\omega t+\phi); \enspace \ddot{x}(t) = - A \omega^2 cos(\omega t+\phi) 
           \end{array}
           \right.\]


\[ \Rightarrow \left \{\begin{array}{ll}
                       A = \frac{f_0}{\sqrt{(\omega_0^2-\omega^2)^2+4\delta^2\omega^2}} \\ 
                       tg(\phi) = - \frac{2\delta\omega}{\omega_0^2-\omega^2)}
           \end{array}
           \right. \]

$ x(t) =
\frac{f_0}{\sqrt{(\omega_0^2-\omega^2)^2+4\delta^2\omega^2}}cos(\omega t+\phi)
\text{,  де} \enspace  tg(\phi) = -
\frac{2\delta\omega}{\omega_0^2-\omega^2)} $

    \subsubsection{Гармонічні
коливання}\label{ux433ux430ux440ux43cux43eux43dux456ux447ux43dux456-ux43aux43eux43bux438ux432ux430ux43dux43dux44f}

    \begin{Verbatim}[commandchars=\\\{\}]
{\color{incolor}In [{\color{incolor}77}]:} \PY{n}{model3}\PY{p}{(}\PY{l+m+mi}{0}\PY{p}{,} \PY{l+m+mf}{0.4}\PY{p}{,} \PY{l+m+mi}{0}\PY{p}{,} \PY{l+m+mi}{0}\PY{p}{,} \PY{l+m+mf}{0.3}\PY{p}{,} \PY{l+m+mf}{0.2}\PY{p}{)}
\end{Verbatim}


    \begin{Verbatim}[commandchars=\\\{\}]
Коефіцієнт згасання: δ = 0
Власна частота: ω0 = 0.4
Частота зовнішньої сили: ω = 0
Амплітуда зовнішньої сили: f0 = 0
Початкове положення: x0 = 0.3
Початкова швидкіть: x0' = 0.2

    \end{Verbatim}

    \begin{center}
    \adjustimage{max size={0.9\linewidth}{0.9\paperheight}}{output_43_1.png}
    \end{center}
    { \hspace*{\fill} \\}
    
    Графік гармонічних коливань (коли \(\delta = 0\), \(f_0 = 0\)).
Амплітуда не міняється з часом, немає ніяких зовнішніх сил.

    \subsubsection{Згасаючі
коливання}\label{ux437ux433ux430ux441ux430ux44eux447ux456-ux43aux43eux43bux438ux432ux430ux43dux43dux44f}

    \begin{Verbatim}[commandchars=\\\{\}]
{\color{incolor}In [{\color{incolor}78}]:} \PY{n}{model3}\PY{p}{(}\PY{l+m+mf}{0.1}\PY{p}{,} \PY{l+m+mf}{0.7}\PY{p}{,} \PY{l+m+mi}{0}\PY{p}{,} \PY{l+m+mi}{0}\PY{p}{,} \PY{l+m+mf}{0.4}\PY{p}{,} \PY{l+m+mf}{1.4}\PY{p}{)}
\end{Verbatim}


    \begin{Verbatim}[commandchars=\\\{\}]
Коефіцієнт згасання: δ = 0.1
Власна частота: ω0 = 0.7
Частота зовнішньої сили: ω = 0
Амплітуда зовнішньої сили: f0 = 0
Початкове положення: x0 = 0.4
Початкова швидкіть: x0' = 1.4

    \end{Verbatim}

    \begin{center}
    \adjustimage{max size={0.9\linewidth}{0.9\paperheight}}{output_46_1.png}
    \end{center}
    { \hspace*{\fill} \\}
    
    Графік згасаючих коливань, коли коефіцієнт згасання \(0 < δ < 1\)
(енергія коливань зменшується з плином часу, наприклад, під дією сил
опору середовища). Зовнішня сила відсутня \(f_0=0\). Амплітуда коливань
в цьому випадку спадає експоненційно:
\(A(t)=C e^{-\zeta \delta _{0}t}\), де \(C\) залежить від початкових
умов.

    \begin{Verbatim}[commandchars=\\\{\}]
{\color{incolor}In [{\color{incolor}79}]:} \PY{n}{model3}\PY{p}{(}\PY{l+m+mi}{1}\PY{p}{,} \PY{l+m+mf}{0.7}\PY{p}{,} \PY{l+m+mi}{0}\PY{p}{,} \PY{l+m+mi}{0}\PY{p}{,} \PY{l+m+mf}{4.1}\PY{p}{,} \PY{o}{\PYZhy{}}\PY{l+m+mf}{0.1}\PY{p}{)}
\end{Verbatim}


    \begin{Verbatim}[commandchars=\\\{\}]
Коефіцієнт згасання: δ = 1
Власна частота: ω0 = 0.7
Частота зовнішньої сили: ω = 0
Амплітуда зовнішньої сили: f0 = 0
Початкове положення: x0 = 4.1
Початкова швидкіть: x0' = -0.1

    \end{Verbatim}

    \begin{center}
    \adjustimage{max size={0.9\linewidth}{0.9\paperheight}}{output_48_1.png}
    \end{center}
    { \hspace*{\fill} \\}
    
    Вище зображено графік згасаючих коливань, коли коефіцієнт згасання
\(δ = 1\), а зовнішня сила відсутня \(f_0=0\). При цьому коливання
відсутні і розв'язок спадає відповідно до наступного закону:
\({\displaystyle x(t)=(c_{1}t+c_{2})e^{-\omega _{o}t}}\), де \(c_1\) і
\(c_2\) знаходяться через початкові умови.

    \begin{Verbatim}[commandchars=\\\{\}]
{\color{incolor}In [{\color{incolor}80}]:} \PY{n}{model3}\PY{p}{(}\PY{l+m+mi}{2}\PY{p}{,} \PY{l+m+mf}{0.7}\PY{p}{,} \PY{l+m+mi}{0}\PY{p}{,} \PY{l+m+mi}{0}\PY{p}{,} \PY{l+m+mf}{4.1}\PY{p}{,} \PY{o}{\PYZhy{}}\PY{l+m+mf}{0.1}\PY{p}{)}
\end{Verbatim}


    \begin{Verbatim}[commandchars=\\\{\}]
Коефіцієнт згасання: δ = 2
Власна частота: ω0 = 0.7
Частота зовнішньої сили: ω = 0
Амплітуда зовнішньої сили: f0 = 0
Початкове положення: x0 = 4.1
Початкова швидкіть: x0' = -0.1

    \end{Verbatim}

    \begin{center}
    \adjustimage{max size={0.9\linewidth}{0.9\paperheight}}{output_50_1.png}
    \end{center}
    { \hspace*{\fill} \\}
    
    Вище зображено графік згасаючих коливань, коли коефіцієнт згасання
\(δ > 1\), а зовнішня сила відсутня \(f_0=0\). При цьому коливання
відсутні і розв'язок експоненційно спадає відповідно до наступного
закону:
\({\displaystyle x(t)=c_{1}e^{\lambda _{-}\,t}+c_{2}e^{\lambda _{+}\,t}}\),
де \(c_1\) і \(c_2\) знаходяться через початкові умови.

    \subsubsection{Вимушені
коливання}\label{ux432ux438ux43cux443ux448ux435ux43dux456-ux43aux43eux43bux438ux432ux430ux43dux43dux44f}

    \begin{Verbatim}[commandchars=\\\{\}]
{\color{incolor}In [{\color{incolor}81}]:} \PY{n}{model3}\PY{p}{(}\PY{l+m+mf}{0.01}\PY{p}{,} \PY{l+m+mf}{0.9}\PY{p}{,} \PY{l+m+mf}{0.8}\PY{p}{,} \PY{l+m+mi}{50}\PY{p}{,} \PY{l+m+mf}{0.4}\PY{p}{,} \PY{l+m+mf}{1.4}\PY{p}{)}
\end{Verbatim}


    \begin{Verbatim}[commandchars=\\\{\}]
Коефіцієнт згасання: δ = 0.01
Власна частота: ω0 = 0.9
Частота зовнішньої сили: ω = 0.8
Амплітуда зовнішньої сили: f0 = 50
Початкове положення: x0 = 0.4
Початкова швидкіть: x0' = 1.4

    \end{Verbatim}

    \begin{center}
    \adjustimage{max size={0.9\linewidth}{0.9\paperheight}}{output_53_1.png}
    \end{center}
    { \hspace*{\fill} \\}
    
    Вимушені коливання при відсутності резонансу, графік розв'язку містить
дві частоти - зовнішню і внутрішню. Вище зображено випадок, коли
\(\omega_0 > \omega\)

    \begin{Verbatim}[commandchars=\\\{\}]
{\color{incolor}In [{\color{incolor}82}]:} \PY{n}{model3}\PY{p}{(}\PY{l+m+mf}{0.01}\PY{p}{,} \PY{l+m+mf}{0.8}\PY{p}{,} \PY{l+m+mf}{1.3}\PY{p}{,} \PY{l+m+mi}{50}\PY{p}{,} \PY{l+m+mf}{0.4}\PY{p}{,} \PY{l+m+mf}{1.4}\PY{p}{)}
\end{Verbatim}


    \begin{Verbatim}[commandchars=\\\{\}]
Коефіцієнт згасання: δ = 0.01
Власна частота: ω0 = 0.8
Частота зовнішньої сили: ω = 1.3
Амплітуда зовнішньої сили: f0 = 50
Початкове положення: x0 = 0.4
Початкова швидкіть: x0' = 1.4

    \end{Verbatim}

    \begin{center}
    \adjustimage{max size={0.9\linewidth}{0.9\paperheight}}{output_55_1.png}
    \end{center}
    { \hspace*{\fill} \\}
    
    Вимушені коливання, коли \(\omega_0 < \omega\)

    \begin{Verbatim}[commandchars=\\\{\}]
{\color{incolor}In [{\color{incolor}83}]:} \PY{n}{model3}\PY{p}{(}\PY{l+m+mf}{0.01}\PY{p}{,} \PY{l+m+mf}{0.7}\PY{p}{,} \PY{l+m+mf}{0.7}\PY{p}{,} \PY{l+m+mi}{50}\PY{p}{,} \PY{l+m+mf}{0.4}\PY{p}{,} \PY{l+m+mf}{1.4}\PY{p}{)}
\end{Verbatim}


    \begin{Verbatim}[commandchars=\\\{\}]
Коефіцієнт згасання: δ = 0.01
Власна частота: ω0 = 0.7
Частота зовнішньої сили: ω = 0.7
Амплітуда зовнішньої сили: f0 = 50
Початкове положення: x0 = 0.4
Початкова швидкіть: x0' = 1.4

    \end{Verbatim}

    \begin{center}
    \adjustimage{max size={0.9\linewidth}{0.9\paperheight}}{output_57_1.png}
    \end{center}
    { \hspace*{\fill} \\}
    
    Явище резонансу при \(\delta > 0\). Причому, максимальна амплітуда
\(A_{max} \approx \frac{\omega_0 f_0}{\delta}\)

    \begin{Verbatim}[commandchars=\\\{\}]
{\color{incolor}In [{\color{incolor}84}]:} \PY{n}{model3}\PY{p}{(}\PY{l+m+mf}{0.}\PY{p}{,} \PY{l+m+mf}{0.7}\PY{p}{,} \PY{l+m+mf}{0.7}\PY{p}{,} \PY{l+m+mi}{50}\PY{p}{,} \PY{l+m+mf}{0.4}\PY{p}{,} \PY{l+m+mf}{1.4}\PY{p}{)}
\end{Verbatim}


    \begin{Verbatim}[commandchars=\\\{\}]
Коефіцієнт згасання: δ = 0.0
Власна частота: ω0 = 0.7
Частота зовнішньої сили: ω = 0.7
Амплітуда зовнішньої сили: f0 = 50
Початкове положення: x0 = 0.4
Початкова швидкіть: x0' = 1.4

    \end{Verbatim}

    \begin{center}
    \adjustimage{max size={0.9\linewidth}{0.9\paperheight}}{output_59_1.png}
    \end{center}
    { \hspace*{\fill} \\}
    
    Явище резонансу при \(\delta = 0\), амплітуда нескінченно зростає.

    \subsection{Коливання у системі
«хижак-жертва»}\label{ux43aux43eux43bux438ux432ux430ux43dux43dux44f-ux443-ux441ux438ux441ux442ux435ux43cux456-ux445ux438ux436ux430ux43a-ux436ux435ux440ux442ux432ux430}

    \[ \left \{\begin{array}{ll}
            x'=(α_x y - β_x)x \enspace - \enspace \text{для популяції хижаків}\\
            y'=(α_y - β_y x)y \enspace - \enspace \text{для популяції жертв}
           \end{array}
           \right. \]

$ α_x $ - "норма споживання" жертв

$ α_y $ - природна народжуваність жертв

$ β_x $ - природна смертність хижаків

$ β_y $ - "норма споживаності" жертв

$ x_0 $ - початкова кількість хижаків

$ y_0 $ - початкова кількість жертв

Характерною особливістю рівннянь є те, що їхнім розв'язком є
автоколивання, тобто амплітуда і період коливань залежать від
властивостей самої системи і не залежать від початкових умов, далі це
покажемо. Тут відбуваються такі процеси: розмноження жертв та їхня
гибель в результаті поїдання хижаками, розмноження та вимирання хижаків.

    Стаціонарні точки:

\[ \left \{\begin{array}{ll}
            x'= 0 \\
            y'= 0
           \end{array}
           \right.  \enspace  \Leftrightarrow \enspace \left \{ \begin{array}{ll}
            (α_x y - β_x)x = 0 \\
            (α_y - β_y x)y = 0
           \end{array}
           \right.  \enspace  \Rightarrow \enspace \left [ \begin{array}{ll}
  \left \{ \begin{array}{ll}
            x^* = 0 \\
            y^* = 0
          \end{array} \right. \\
  \left \{ \begin{array}{ll}
            x^* = \frac{α_y}{β_y}\\
            y^* = \frac{β_x}{α_x}
           \end{array} \right. \\
           \end{array}
           \right. \]

    Стаціонарна точка $ (x^*, y^*) = (\frac{α_y}{β_y},\frac{β_x}{α_x}) \enspace - $ положення рівноваги, в якому чисельність
жертв і хижаків залишаються сталими. З відхиленням від цієї точки
спостерігаються коливальні процеси зміни популяцій.

    Коливання:

Дослідимо систему при незначних відхиленнях від $ (x^*, y^*) $:

Нехай $\enspace x = x^* + u, y = y^* + v $; $ \enspace u,v $ -
відносно маленькі величини

$ \dot{u} =\dot{x}=(α_xv + α_x y^* - β_x) (x^*+u)=α_xv(x^*+u) \approx α_xvx^* = \frac{α_xα_y}{β_y}v $

$ \dot{v} = \dot{x}=(α_y - β_y x^* - β_yu)(y^*+v) = (-β_yu)(y^*+v) \approx - β_yuy^* = - \frac{β_x β_y}{α_x}u $

$ \Rightarrow \enspace \ddot{u} = \frac{α_xα_y}{β_y}\dot{v} = -α_yβ_xu \enspace \Rightarrow \enspace \ddot{u} + α_yβ_xu = 0$

$ \enspace  \enspace \enspace \ddot{v} = - \frac{β_x β_y}{α_x}\dot{u} =- α_yβ_xv \enspace \Rightarrow \enspace \ddot{v} + α_yβ_xv = 0$

$ \Rightarrow \omega_0 = \sqrt{α_yβ_x} \enspace \text{як для u, так і для v}$

Отже, частота коливань для обох популяцій однакова і визначається
властивостями системи - природною народжуваністю жертв і природною
смертністю хижаків

Амплітуди і фази коливань залежать від початкових умов.

    \subsubsection{Стаціонарний
процес}\label{ux441ux442ux430ux446ux456ux43eux43dux430ux440ux43dux438ux439-ux43fux440ux43eux446ux435ux441}

    \begin{Verbatim}[commandchars=\\\{\}]
{\color{incolor}In [{\color{incolor}86}]:} \PY{n}{model4}\PY{p}{(}\PY{l+m+mf}{0.6}\PY{p}{,} \PY{l+m+mf}{0.7}\PY{p}{,} \PY{l+m+mf}{0.5}\PY{p}{,} \PY{l+m+mf}{0.6}\PY{p}{,} \PY{l+m+mi}{0}\PY{p}{,} \PY{l+m+mi}{0}\PY{p}{)}
\end{Verbatim}


    \begin{Verbatim}[commandchars=\\\{\}]
"Норма споживання" жертв: αx = 0.6
Природна народжуваність жертв: αy = 0.7
Природна смертність хижаків: βx = 0.5
"Норма споживаності" жертв: βy = 0.6
Початкова кількість хижаків: x0 = 0
Початкова кількість жертв: y0 = 0

    \end{Verbatim}

    \begin{center}
    \adjustimage{max size={0.9\linewidth}{0.9\paperheight}}{output_67_1.png}
    \end{center}
    { \hspace*{\fill} \\}
    
    Тривіальний стаціонарний розв'язок:
\(\left \{ \begin{array}{ll}  x_0 = 0 \\  y_0 = 0  \end{array} \right.\)
тобто в популяції немає ні жертв, ні хижаків.

    \begin{Verbatim}[commandchars=\\\{\}]
{\color{incolor}In [{\color{incolor}87}]:} \PY{n}{αx} \PY{o}{=} \PY{l+m+mf}{0.3}
         \PY{n}{αy} \PY{o}{=} \PY{l+m+mf}{0.3}
         \PY{n}{βx} \PY{o}{=} \PY{l+m+mf}{0.3}
         \PY{n}{βy} \PY{o}{=} \PY{l+m+mf}{0.6}
         \PY{n}{model4}\PY{p}{(}\PY{n}{αx}\PY{p}{,} \PY{n}{αy}\PY{p}{,} \PY{n}{βx}\PY{p}{,} \PY{n}{βy}\PY{p}{,} \PY{n}{αy}\PY{o}{/}\PY{n}{βy}\PY{p}{,} \PY{n}{βx}\PY{o}{/}\PY{n}{αx}\PY{p}{)}
\end{Verbatim}


    \begin{Verbatim}[commandchars=\\\{\}]
"Норма споживання" жертв: αx = 0.3
Природна народжуваність жертв: αy = 0.3
Природна смертність хижаків: βx = 0.3
"Норма споживаності" жертв: βy = 0.6
Початкова кількість хижаків: x0 = 0.5
Початкова кількість жертв: y0 = 1.0

    \end{Verbatim}

    \begin{center}
    \adjustimage{max size={0.9\linewidth}{0.9\paperheight}}{output_69_1.png}
    \end{center}
    { \hspace*{\fill} \\}
    
    Стаціонарний розв'язок, коли чисельності популяцій не змінюються: 
\[ \left \{ \begin{array}{ll}
            x_0 = \frac{α_y}{β_y}\\
            y_0 = \frac{β_x}{α_x}
           \end{array}
           \right. \]

    \subsubsection{Коливання при незначних відхиленнях початкових умов від
рівноваги}\label{ux43aux43eux43bux438ux432ux430ux43dux43dux44f-ux43fux440ux438-ux43dux435ux437ux43dux430ux447ux43dux438ux445-ux432ux456ux434ux445ux438ux43bux435ux43dux43dux44fux445-ux43fux43eux447ux430ux442ux43aux43eux432ux438ux445-ux443ux43cux43eux432-ux432ux456ux434-ux440ux456ux432ux43dux43eux432ux430ux433ux438}

    \begin{Verbatim}[commandchars=\\\{\}]
{\color{incolor}In [{\color{incolor}88}]:} \PY{n}{model4}\PY{p}{(}\PY{l+m+mf}{0.5}\PY{p}{,} \PY{l+m+mf}{0.3}\PY{p}{,} \PY{l+m+mf}{0.5}\PY{p}{,} \PY{l+m+mf}{0.6}\PY{p}{,} \PY{l+m+mf}{0.5005}\PY{p}{,} \PY{l+m+mf}{1.005}\PY{p}{)}
\end{Verbatim}


    \begin{Verbatim}[commandchars=\\\{\}]
"Норма споживання" жертв: αx = 0.5
Природна народжуваність жертв: αy = 0.3
Природна смертність хижаків: βx = 0.5
"Норма споживаності" жертв: βy = 0.6
Початкова кількість хижаків: x0 = 0.5005
Початкова кількість жертв: y0 = 1.005

    \end{Verbatim}

    \begin{center}
    \adjustimage{max size={0.9\linewidth}{0.9\paperheight}}{output_72_1.png}
    \end{center}
    { \hspace*{\fill} \\}
    
    Розв'язком завжди буде періодична функція. Фазові траєкторії ---
замкнуті криві, всередині яких знаходиться фокус (точка рівноваги), яку
позначено фіолетовим. При незначних відхиленнях фігура схожа на еліпс,
але не є еліпсом (при відхиленнях, прямуючих до нуля, фігура прямуватиме
до еліпса)

    \subsubsection{Коливання при значних відхиленнях початкових умов від
рівноваги}\label{ux43aux43eux43bux438ux432ux430ux43dux43dux44f-ux43fux440ux438-ux437ux43dux430ux447ux43dux438ux445-ux432ux456ux434ux445ux438ux43bux435ux43dux43dux44fux445-ux43fux43eux447ux430ux442ux43aux43eux432ux438ux445-ux443ux43cux43eux432-ux432ux456ux434-ux440ux456ux432ux43dux43eux432ux430ux433ux438}

    \begin{Verbatim}[commandchars=\\\{\}]
{\color{incolor}In [{\color{incolor}89}]:} \PY{n}{model4}\PY{p}{(}\PY{l+m+mf}{0.6}\PY{p}{,} \PY{l+m+mf}{0.2}\PY{p}{,} \PY{l+m+mf}{0.5}\PY{p}{,} \PY{l+m+mf}{0.6}\PY{p}{,} \PY{l+m+mf}{0.9}\PY{p}{,} \PY{l+m+mf}{1.4}\PY{p}{)}
\end{Verbatim}


    \begin{Verbatim}[commandchars=\\\{\}]
"Норма споживання" жертв: αx = 0.6
Природна народжуваність жертв: αy = 0.2
Природна смертність хижаків: βx = 0.5
"Норма споживаності" жертв: βy = 0.6
Початкова кількість хижаків: x0 = 0.9
Початкова кількість жертв: y0 = 1.4

    \end{Verbatim}

    \begin{center}
    \adjustimage{max size={0.9\linewidth}{0.9\paperheight}}{output_75_1.png}
    \end{center}
    { \hspace*{\fill} \\}
    
    Можна помітити, що кількість хижаків - періодично змінються із
запізненням відносно до кількості жертв. Це підтверджує той факт, що
розмноження хижаків залежить прямо пропорційно від кількості іжі, тобто,
кількості потенційних жертв у популяції. Дійсно, при значному
розмноженні жертв створюються умови для розмноження хижаків завдяки
доступності їжі. Але розмноження хижаків призводить до зменшення числа
жертв. Коли число жертв сильно падає, хижаки теж гинуть через недостатню
кількість їжі. Тільки тоді, коли кількість хижаків досягає мінімуму,
популяція жертв знову починає зростати.

    \begin{Verbatim}[commandchars=\\\{\}]
{\color{incolor}In [{\color{incolor}90}]:} \PY{n}{model4}\PY{p}{(}\PY{l+m+mf}{0.8}\PY{p}{,} \PY{l+m+mf}{0.5}\PY{p}{,} \PY{l+m+mf}{0.7}\PY{p}{,} \PY{l+m+mf}{0.1}\PY{p}{,} \PY{l+m+mf}{5.2}\PY{p}{,} \PY{l+m+mf}{4.5}\PY{p}{)}
\end{Verbatim}


    \begin{Verbatim}[commandchars=\\\{\}]
"Норма споживання" жертв: αx = 0.8
Природна народжуваність жертв: αy = 0.5
Природна смертність хижаків: βx = 0.7
"Норма споживаності" жертв: βy = 0.1
Початкова кількість хижаків: x0 = 5.2
Початкова кількість жертв: y0 = 4.5

    \end{Verbatim}

    \begin{center}
    \adjustimage{max size={0.9\linewidth}{0.9\paperheight}}{output_77_1.png}
    \end{center}
    { \hspace*{\fill} \\}
    
    В даному випадку кількість хижаків набагато більша. На це дуже впливає
норма споживання жертв і норма споживаності жертв - тобто те, скільки
потенційно може з'явитися хижаків від поїдання однієї жертви, та те,
скільки жертв потенційно вимре від одного хижака відповідно. В даному
випадку перший параметр досить високий, а другий досить низький. Також
слід відмітити досить високу народжуваність жертв. Це все призводить до
такої картини.

    \newpage
    \section{Висновки}\label{ux432ux438ux441ux43dux43eux432ux43aux438}

    В даній роботі ми дослідили чотири процеси, аналітично та
експериментально знайшли в них стаціонарні точки, точки перегину та інші
різні характеристики типу амплітуди, частоти коливань; проілюстрували та
проаналізували важливі типи розв'язків.

    \newpage
    \section{Код
програми}\label{ux43aux43eux434-ux43fux440ux43eux433ux440ux430ux43cux438}

    \begin{Verbatim}[commandchars=\\\{\}]
{\color{incolor}In [{\color{incolor}70}]:} \PY{k+kn}{from} \PY{n+nn}{ipywidgets} \PY{k}{import} \PY{o}{*}
         \PY{k+kn}{import} \PY{n+nn}{numpy} \PY{k}{as} \PY{n+nn}{np}
         \PY{k+kn}{from} \PY{n+nn}{scipy}\PY{n+nn}{.}\PY{n+nn}{integrate} \PY{k}{import} \PY{n}{odeint}
         \PY{k+kn}{import} \PY{n+nn}{matplotlib}\PY{n+nn}{.}\PY{n+nn}{pyplot} \PY{k}{as} \PY{n+nn}{plt}
         \PY{k+kn}{from} \PY{n+nn}{IPython}\PY{n+nn}{.}\PY{n+nn}{display} \PY{k}{import} \PY{n}{display}
         \PY{o}{\PYZpc{}}\PY{k}{matplotlib} inline
         
         \PY{k+kn}{from} \PY{n+nn}{pylab} \PY{k}{import} \PY{n}{rcParams}
         \PY{n}{rcParams}\PY{p}{[}\PY{l+s+s1}{\PYZsq{}}\PY{l+s+s1}{figure.figsize}\PY{l+s+s1}{\PYZsq{}}\PY{p}{]} \PY{o}{=} \PY{l+m+mi}{14}\PY{p}{,} \PY{l+m+mi}{6}
         \PY{n}{style} \PY{o}{=} \PY{p}{\PYZob{}}\PY{l+s+s1}{\PYZsq{}}\PY{l+s+s1}{description\PYZus{}width}\PY{l+s+s1}{\PYZsq{}}\PY{p}{:} \PY{l+s+s1}{\PYZsq{}}\PY{l+s+s1}{initial}\PY{l+s+s1}{\PYZsq{}}\PY{p}{\PYZcb{}}
         \PY{n}{layout} \PY{o}{=} \PY{n}{Layout}\PY{p}{(}\PY{n}{width} \PY{o}{=} \PY{l+s+s1}{\PYZsq{}}\PY{l+s+s1}{400px}\PY{l+s+s1}{\PYZsq{}}\PY{p}{)}
         \PY{n}{models} \PY{o}{=} \PY{p}{[}\PY{k+kc}{None}\PY{p}{]} \PY{o}{*} \PY{l+m+mi}{4} 
         
         
         
         \PY{k}{def} \PY{n+nf}{model1}\PY{p}{(}\PY{n}{k}\PY{p}{,} \PY{n}{μ}\PY{p}{,} \PY{n}{N0}\PY{p}{)}\PY{p}{:}
             \PY{k}{def} \PY{n+nf}{model1\PYZus{}eq}\PY{p}{(}\PY{n}{N}\PY{p}{,}\PY{n}{t}\PY{p}{)}\PY{p}{:}
                 \PY{k}{return} \PY{n}{μ}\PY{o}{*}\PY{n}{N}\PY{o}{*}\PY{p}{(}\PY{n}{k}\PY{o}{\PYZhy{}}\PY{n}{N}\PY{p}{)}
             
             \PY{n}{t} \PY{o}{=} \PY{n}{np}\PY{o}{.}\PY{n}{linspace}\PY{p}{(}\PY{l+m+mi}{0}\PY{p}{,} \PY{l+m+mi}{30}\PY{p}{,} \PY{n}{num}\PY{o}{=}\PY{l+m+mi}{200}\PY{p}{)}
         
             \PY{c+c1}{\PYZsh{} solve ODE}
             \PY{n}{N} \PY{o}{=} \PY{n}{odeint}\PY{p}{(}\PY{n}{model1\PYZus{}eq}\PY{p}{,} \PY{n}{N0}\PY{p}{,} \PY{n}{t}\PY{p}{)}
         
             \PY{n+nb}{print}\PY{p}{(}\PY{l+s+s1}{\PYZsq{}}\PY{l+s+s1}{Ємність середовища (гранична чисельність): k = }\PY{l+s+si}{\PYZob{}\PYZcb{}}\PY{l+s+s1}{\PYZsq{}}\PY{o}{.}\PY{n}{format}\PY{p}{(}\PY{n}{k}\PY{p}{)}\PY{p}{)}
             \PY{n+nb}{print}\PY{p}{(}\PY{l+s+s1}{\PYZsq{}}\PY{l+s+s1}{Швидкість росту популяції: μ = }\PY{l+s+si}{\PYZob{}\PYZcb{}}\PY{l+s+s1}{\PYZsq{}}\PY{o}{.}\PY{n}{format}\PY{p}{(}\PY{n}{μ}\PY{p}{)}\PY{p}{)}
             \PY{n+nb}{print}\PY{p}{(}\PY{l+s+s1}{\PYZsq{}}\PY{l+s+s1}{Початковий розмір популяції: N0 = }\PY{l+s+si}{\PYZob{}\PYZcb{}}\PY{l+s+s1}{\PYZsq{}}\PY{o}{.}\PY{n}{format}\PY{p}{(}\PY{n}{N0}\PY{p}{)}\PY{p}{)}
             
             \PY{c+c1}{\PYZsh{} plot results}
             \PY{n}{plt}\PY{o}{.}\PY{n}{plot}\PY{p}{(}\PY{n}{t}\PY{p}{,} \PY{n}{N}\PY{p}{)}
             \PY{n}{plt}\PY{o}{.}\PY{n}{xlabel}\PY{p}{(}\PY{l+s+s1}{\PYZsq{}}\PY{l+s+s1}{t}\PY{l+s+s1}{\PYZsq{}}\PY{p}{)}
             \PY{n}{plt}\PY{o}{.}\PY{n}{ylabel}\PY{p}{(}\PY{l+s+s1}{\PYZsq{}}\PY{l+s+s1}{N(t)}\PY{l+s+s1}{\PYZsq{}}\PY{p}{)}
             \PY{n}{plt}\PY{o}{.}\PY{n}{show}\PY{p}{(}\PY{p}{)}
         
         \PY{k}{def} \PY{n+nf}{plot\PYZus{}model1}\PY{p}{(}\PY{n}{k}\PY{p}{,} \PY{n}{mu}\PY{p}{,} \PY{n}{N0}\PY{p}{)}\PY{p}{:}
             \PY{n}{description1} \PY{o}{=} \PY{n}{Label}\PY{p}{(}\PY{n}{value}\PY{o}{=}\PY{l+s+s2}{\PYZdq{}}\PY{l+s+s2}{\PYZdl{}N}\PY{l+s+s2}{\PYZsq{}}\PY{l+s+s2}{=μN(k\PYZhy{}N)\PYZdl{}}\PY{l+s+s2}{\PYZdq{}}\PY{p}{)}
             \PY{n}{k\PYZus{}val} \PY{o}{=} \PY{n}{FloatText}\PY{p}{(}\PY{n}{value}\PY{o}{=}\PY{n}{k}\PY{p}{,} \PY{n}{description} \PY{o}{=} \PYZbs{}
             \PY{l+s+s1}{\PYZsq{}}\PY{l+s+s1}{Ємність середовища (гранична чисельність): \PYZdl{}k = \PYZdl{}}\PY{l+s+s1}{\PYZsq{}}\PY{p}{,} \PY{n}{step}\PY{o}{=}\PY{l+m+mf}{0.1}\PY{p}{,}
                               \PY{n}{style} \PY{o}{=} \PY{n}{style}\PY{p}{,} \PY{n}{layout} \PY{o}{=} \PY{n}{layout}\PY{p}{)}
             \PY{n}{μ\PYZus{}val} \PY{o}{=} \PY{n}{FloatText}\PY{p}{(}\PY{n}{value}\PY{o}{=}\PY{n}{mu}\PY{p}{,} \PY{n}{description} \PY{o}{=} \PYZbs{}
             \PY{l+s+s1}{\PYZsq{}}\PY{l+s+s1}{Швидкість росту популяції: \PYZdl{}μ = \PYZdl{}}\PY{l+s+s1}{\PYZsq{}}\PY{p}{,} \PY{n}{step}\PY{o}{=}\PY{l+m+mf}{0.1}\PY{p}{,} 
                               \PY{n}{style} \PY{o}{=} \PY{n}{style}\PY{p}{,} \PY{n}{layout} \PY{o}{=} \PY{n}{layout}\PY{p}{)}
             \PY{n}{N0\PYZus{}val} \PY{o}{=} \PY{n}{FloatText}\PY{p}{(}\PY{n}{value}\PY{o}{=}\PY{n}{N0}\PY{p}{,} \PY{n}{description} \PY{o}{=} \PYZbs{}
             \PY{l+s+s1}{\PYZsq{}}\PY{l+s+s1}{Початковий розмір популяції: \PYZdl{}N\PYZus{}0 = \PYZdl{}}\PY{l+s+s1}{\PYZsq{}}\PY{p}{,} \PY{n}{step}\PY{o}{=}\PY{l+m+mf}{0.1}\PY{p}{,}
                                \PY{n}{style} \PY{o}{=} \PY{n}{style}\PY{p}{,} \PY{n}{layout} \PY{o}{=} \PY{n}{layout}\PY{p}{)}
             \PY{n}{models}\PY{p}{[}\PY{l+m+mi}{0}\PY{p}{]} \PY{o}{=} \PY{n}{VBox}\PY{p}{(}\PY{p}{[}\PY{n}{description1}\PY{p}{,} \PY{n}{interactive}\PY{p}{(}\PY{n}{model1}\PY{p}{,} \PY{n}{k}\PY{o}{=}\PY{n}{k\PYZus{}val}\PY{p}{,}
                                                         \PY{n}{μ}\PY{o}{=}\PY{n}{μ\PYZus{}val}\PY{p}{,} \PY{n}{N0}\PY{o}{=}\PY{n}{N0\PYZus{}val}\PY{p}{,}
                                                         \PY{n}{continuous\PYZus{}update}\PY{o}{=}\PY{k+kc}{True}\PY{p}{)}\PY{p}{]}\PY{p}{)}
             \PY{n}{display}\PY{p}{(}\PY{n}{models}\PY{p}{[}\PY{l+m+mi}{0}\PY{p}{]}\PY{p}{)}
             
         
         \PY{k}{def} \PY{n+nf}{model2}\PY{p}{(}\PY{n}{s}\PY{p}{,} \PY{n}{a}\PY{p}{,} \PY{n}{α}\PY{p}{,} \PY{n}{μ}\PY{p}{,} \PY{n}{q}\PY{p}{,} \PY{n}{k0}\PY{p}{)}\PY{p}{:}
             \PY{k}{def} \PY{n+nf}{model2\PYZus{}eq}\PY{p}{(}\PY{n}{k}\PY{p}{,}\PY{n}{t}\PY{p}{)}\PY{p}{:}
                 \PY{k}{return} \PY{n}{s}\PY{o}{*}\PY{n}{a}\PY{o}{*}\PY{n}{k}\PY{o}{*}\PY{o}{*}\PY{n}{α}\PY{o}{\PYZhy{}}\PY{p}{(}\PY{n}{μ}\PY{o}{+}\PY{n}{q}\PY{p}{)}\PY{o}{*}\PY{n}{k}
         
             \PY{n}{t} \PY{o}{=} \PY{n}{np}\PY{o}{.}\PY{n}{linspace}\PY{p}{(}\PY{l+m+mi}{0}\PY{p}{,} \PY{l+m+mi}{100}\PY{p}{,} \PY{n}{num}\PY{o}{=}\PY{l+m+mi}{200}\PY{p}{)}
         
             \PY{c+c1}{\PYZsh{} solve ODE}
             \PY{n}{k} \PY{o}{=} \PY{n}{odeint}\PY{p}{(}\PY{n}{model2\PYZus{}eq}\PY{p}{,} \PY{n}{k0}\PY{p}{,} \PY{n}{t}\PY{p}{)}
         
             \PY{n+nb}{print}\PY{p}{(}\PY{l+s+s1}{\PYZsq{}}\PY{l+s+s1}{Норма накопичення: s = }\PY{l+s+si}{\PYZob{}\PYZcb{}}\PY{l+s+s1}{\PYZsq{}}\PY{o}{.}\PY{n}{format}\PY{p}{(}\PY{n}{s}\PY{p}{)}\PY{p}{)}
             \PY{n+nb}{print}\PY{p}{(}\PY{l+s+s1}{\PYZsq{}}\PY{l+s+s1}{Рівень розвитку економіки: a = }\PY{l+s+si}{\PYZob{}\PYZcb{}}\PY{l+s+s1}{\PYZsq{}}\PY{o}{.}\PY{n}{format}\PY{p}{(}\PY{n}{a}\PY{p}{)}\PY{p}{)}
             \PY{n+nb}{print}\PY{p}{(}\PY{l+s+s1}{\PYZsq{}}\PY{l+s+s1}{Частка капіталу у сукупній продукції: α = }\PY{l+s+si}{\PYZob{}\PYZcb{}}\PY{l+s+s1}{\PYZsq{}}\PY{o}{.}\PY{n}{format}\PY{p}{(}\PY{n}{α}\PY{p}{)}\PY{p}{)}
             \PY{n+nb}{print}\PY{p}{(}\PY{l+s+s1}{\PYZsq{}}\PY{l+s+s1}{Норма витрат на амортизацію: μ = }\PY{l+s+si}{\PYZob{}\PYZcb{}}\PY{l+s+s1}{\PYZsq{}}\PY{o}{.}\PY{n}{format}\PY{p}{(}\PY{n}{μ}\PY{p}{)}\PY{p}{)}
             \PY{n+nb}{print}\PY{p}{(}\PY{l+s+s1}{\PYZsq{}}\PY{l+s+s1}{Темп приросту робочої сили: q = }\PY{l+s+si}{\PYZob{}\PYZcb{}}\PY{l+s+s1}{\PYZsq{}}\PY{o}{.}\PY{n}{format}\PY{p}{(}\PY{n}{q}\PY{p}{)}\PY{p}{)}
             \PY{n+nb}{print}\PY{p}{(}\PY{l+s+s1}{\PYZsq{}}\PY{l+s+s1}{Початкова капіталозабезпеченість: k0 = }\PY{l+s+si}{\PYZob{}\PYZcb{}}\PY{l+s+s1}{\PYZsq{}}\PY{o}{.}\PY{n}{format}\PY{p}{(}\PY{n}{k0}\PY{p}{)}\PY{p}{)}
             
             \PY{c+c1}{\PYZsh{} plot results}
             \PY{n}{plt}\PY{o}{.}\PY{n}{plot}\PY{p}{(}\PY{n}{t}\PY{p}{,} \PY{n}{k}\PY{p}{)}
             \PY{n}{plt}\PY{o}{.}\PY{n}{xlabel}\PY{p}{(}\PY{l+s+s1}{\PYZsq{}}\PY{l+s+s1}{t}\PY{l+s+s1}{\PYZsq{}}\PY{p}{)}
             \PY{n}{plt}\PY{o}{.}\PY{n}{ylabel}\PY{p}{(}\PY{l+s+s1}{\PYZsq{}}\PY{l+s+s1}{k(t)}\PY{l+s+s1}{\PYZsq{}}\PY{p}{)}
             \PY{n}{plt}\PY{o}{.}\PY{n}{show}\PY{p}{(}\PY{p}{)}
         
         \PY{k}{def} \PY{n+nf}{plot\PYZus{}model2}\PY{p}{(}\PY{n}{s}\PY{p}{,} \PY{n}{a}\PY{p}{,} \PY{n}{alpha}\PY{p}{,} \PY{n}{mu}\PY{p}{,} \PY{n}{q}\PY{p}{,} \PY{n}{k0}\PY{p}{)}\PY{p}{:}
             \PY{n}{description2} \PY{o}{=} \PY{n}{Label}\PY{p}{(}\PY{n}{value}\PY{o}{=}\PY{l+s+s2}{\PYZdq{}}\PY{l+s+s2}{\PYZdl{}k}\PY{l+s+s2}{\PYZsq{}}\PY{l+s+s2}{=sak\PYZca{}α\PYZhy{}(μ+q)k\PYZdl{}}\PY{l+s+s2}{\PYZdq{}}\PY{p}{)}
             \PY{n}{s\PYZus{}val} \PY{o}{=} \PY{n}{FloatText}\PY{p}{(}\PY{n}{s}\PY{p}{,} \PY{n}{description} \PY{o}{=} \PYZbs{}
             \PY{l+s+s1}{\PYZsq{}}\PY{l+s+s1}{Норма накопичення: \PYZdl{}s = \PYZdl{}}\PY{l+s+s1}{\PYZsq{}}\PY{p}{,} \PY{n}{step}\PY{o}{=}\PY{l+m+mf}{0.1}\PY{p}{,} \PY{n}{style} \PY{o}{=} \PY{n}{style}\PY{p}{,} \PY{n}{layout} \PY{o}{=} \PY{n}{layout}\PY{p}{)}
             \PY{n}{a\PYZus{}val} \PY{o}{=} \PY{n}{FloatText}\PY{p}{(}\PY{n}{a}\PY{p}{,} \PY{n}{description} \PY{o}{=} \PYZbs{}
             \PY{l+s+s1}{\PYZsq{}}\PY{l+s+s1}{Рівень розвитку економіки: \PYZdl{}a = \PYZdl{}}\PY{l+s+s1}{\PYZsq{}}\PY{p}{,} \PY{n}{step}\PY{o}{=}\PY{l+m+mf}{0.1}\PY{p}{,}
                               \PY{n}{style} \PY{o}{=} \PY{n}{style}\PY{p}{,} \PY{n}{layout} \PY{o}{=} \PY{n}{layout}\PY{p}{)}
             \PY{n}{α\PYZus{}val} \PY{o}{=} \PY{n}{FloatText}\PY{p}{(}\PY{n}{alpha}\PY{p}{,} \PY{n}{description} \PY{o}{=} \PYZbs{}
             \PY{l+s+s1}{\PYZsq{}}\PY{l+s+s1}{Частка капіталу у сукупній продукції: \PYZdl{}α = \PYZdl{}}\PY{l+s+s1}{\PYZsq{}}\PY{p}{,} \PY{n}{step}\PY{o}{=}\PY{l+m+mf}{0.1}\PY{p}{,}
                               \PY{n}{style} \PY{o}{=} \PY{n}{style}\PY{p}{,} \PY{n}{layout} \PY{o}{=} \PY{n}{layout}\PY{p}{)}
             \PY{n}{μ\PYZus{}val} \PY{o}{=} \PY{n}{FloatText}\PY{p}{(}\PY{n}{mu}\PY{p}{,} \PY{n}{description} \PY{o}{=} \PYZbs{}
             \PY{l+s+s1}{\PYZsq{}}\PY{l+s+s1}{Норма витрат на амортизацію: \PYZdl{}μ = \PYZdl{}}\PY{l+s+s1}{\PYZsq{}}\PY{p}{,} \PY{n}{step}\PY{o}{=}\PY{l+m+mf}{0.1}\PY{p}{,}
                               \PY{n}{style} \PY{o}{=} \PY{n}{style}\PY{p}{,} \PY{n}{layout} \PY{o}{=} \PY{n}{layout}\PY{p}{)}
             \PY{n}{q\PYZus{}val} \PY{o}{=} \PY{n}{FloatText}\PY{p}{(}\PY{n}{q}\PY{p}{,} \PY{n}{description} \PY{o}{=} \PYZbs{}
             \PY{l+s+s1}{\PYZsq{}}\PY{l+s+s1}{Темп приросту робочої сили: \PYZdl{}q = \PYZdl{}}\PY{l+s+s1}{\PYZsq{}}\PY{p}{,} \PY{n}{step}\PY{o}{=}\PY{l+m+mf}{0.1}\PY{p}{,}
                               \PY{n}{style} \PY{o}{=} \PY{n}{style}\PY{p}{,} \PY{n}{layout} \PY{o}{=} \PY{n}{layout}\PY{p}{)}
             \PY{n}{k0\PYZus{}val} \PY{o}{=} \PY{n}{FloatText}\PY{p}{(}\PY{n}{k0}\PY{p}{,} \PY{n}{description} \PY{o}{=} \PYZbs{}
             \PY{l+s+s1}{\PYZsq{}}\PY{l+s+s1}{Початкова капіталозабезпеченість: \PYZdl{}k\PYZus{}0 = \PYZdl{}}\PY{l+s+s1}{\PYZsq{}}\PY{p}{,} \PY{n}{step}\PY{o}{=}\PY{l+m+mf}{0.1}\PY{p}{,}
                                \PY{n}{style} \PY{o}{=} \PY{n}{style}\PY{p}{,} \PY{n}{layout} \PY{o}{=} \PY{n}{layout}\PY{p}{)}
             \PY{n}{models}\PY{p}{[}\PY{l+m+mi}{1}\PY{p}{]} \PY{o}{=} \PY{n}{VBox}\PY{p}{(}\PY{p}{[}\PY{n}{description2}\PY{p}{,} \PY{n}{interactive}\PY{p}{(}\PY{n}{model2}\PY{p}{,} \PY{n}{s}\PY{o}{=}\PY{n}{s\PYZus{}val}\PY{p}{,}
                                                         \PY{n}{a}\PY{o}{=}\PY{n}{a\PYZus{}val}\PY{p}{,} \PY{n}{α}\PY{o}{=}\PY{n}{α\PYZus{}val}\PY{p}{,}
                                                         \PY{n}{μ}\PY{o}{=}\PY{n}{μ\PYZus{}val}\PY{p}{,} \PY{n}{q}\PY{o}{=}\PY{n}{q\PYZus{}val}\PY{p}{,} \PY{n}{k0}\PY{o}{=}\PY{n}{k0\PYZus{}val}\PY{p}{,}
                                                         \PY{n}{continuous\PYZus{}update}\PY{o}{=}\PY{k+kc}{True}\PY{p}{)}\PY{p}{]}\PY{p}{)}
             \PY{n}{display}\PY{p}{(}\PY{n}{models}\PY{p}{[}\PY{l+m+mi}{1}\PY{p}{]}\PY{p}{)}
             
             
         \PY{k}{def} \PY{n+nf}{model3}\PY{p}{(}\PY{n}{δ}\PY{p}{,} \PY{n}{ω0}\PY{p}{,} \PY{n}{ω}\PY{p}{,} \PY{n}{f0}\PY{p}{,} \PY{n}{x0}\PY{p}{,} \PY{n}{x00}\PY{p}{)}\PY{p}{:}
             
             \PY{c+c1}{\PYZsh{}x\PYZus{}0\PYZsq{} = x\PYZus{}1 = x\PYZsq{}}
             \PY{c+c1}{\PYZsh{}x\PYZus{}1\PYZsq{} = x\PYZsq{}\PYZsq{} = f0 * np.cos(ω * t) \PYZhy{} 2 * δ * x[1] \PYZhy{} (ω0 ** 2) * x[0]}
             \PY{k}{def} \PY{n+nf}{model3\PYZus{}eq}\PY{p}{(}\PY{n}{x}\PY{p}{,}\PY{n}{t}\PY{p}{)}\PY{p}{:}
                 \PY{k}{return} \PY{p}{[}\PY{n}{x}\PY{p}{[}\PY{l+m+mi}{1}\PY{p}{]}\PY{p}{,} \PY{n}{f0} \PY{o}{*} \PY{n}{np}\PY{o}{.}\PY{n}{cos}\PY{p}{(}\PY{n}{ω} \PY{o}{*} \PY{n}{t}\PY{p}{)} \PY{o}{\PYZhy{}} \PY{l+m+mi}{2} \PY{o}{*} \PY{n}{δ} \PY{o}{*} \PY{n}{x}\PY{p}{[}\PY{l+m+mi}{1}\PY{p}{]} \PY{o}{\PYZhy{}} \PY{p}{(}\PY{n}{ω0} \PY{o}{*}\PY{o}{*} \PY{l+m+mi}{2}\PY{p}{)} \PY{o}{*} \PY{n}{x}\PY{p}{[}\PY{l+m+mi}{0}\PY{p}{]}\PY{p}{]}
         
             \PY{n}{t} \PY{o}{=} \PY{n}{np}\PY{o}{.}\PY{n}{linspace}\PY{p}{(}\PY{l+m+mi}{0}\PY{p}{,} \PY{l+m+mi}{300}\PY{p}{,} \PY{n}{num}\PY{o}{=}\PY{l+m+mi}{1000}\PY{p}{)}
         
             \PY{c+c1}{\PYZsh{} solve ODE}
             \PY{n}{x} \PY{o}{=} \PY{n}{odeint}\PY{p}{(}\PY{n}{model3\PYZus{}eq}\PY{p}{,} \PY{n}{np}\PY{o}{.}\PY{n}{array}\PY{p}{(}\PY{p}{[}\PY{n}{x0}\PY{p}{,} \PY{n}{x00}\PY{p}{]}\PY{p}{)}\PY{p}{,} \PY{n}{t}\PY{p}{)}
         
             \PY{n+nb}{print}\PY{p}{(}\PY{l+s+s1}{\PYZsq{}}\PY{l+s+s1}{Коефіцієнт згасання: δ = }\PY{l+s+si}{\PYZob{}\PYZcb{}}\PY{l+s+s1}{\PYZsq{}}\PY{o}{.}\PY{n}{format}\PY{p}{(}\PY{n}{δ}\PY{p}{)}\PY{p}{)}
             \PY{n+nb}{print}\PY{p}{(}\PY{l+s+s1}{\PYZsq{}}\PY{l+s+s1}{Власна частота: ω0 = }\PY{l+s+si}{\PYZob{}\PYZcb{}}\PY{l+s+s1}{\PYZsq{}}\PY{o}{.}\PY{n}{format}\PY{p}{(}\PY{n}{ω0}\PY{p}{)}\PY{p}{)}
             \PY{n+nb}{print}\PY{p}{(}\PY{l+s+s1}{\PYZsq{}}\PY{l+s+s1}{Частота зовнішньої сили: ω = }\PY{l+s+si}{\PYZob{}\PYZcb{}}\PY{l+s+s1}{\PYZsq{}}\PY{o}{.}\PY{n}{format}\PY{p}{(}\PY{n}{ω}\PY{p}{)}\PY{p}{)}
             \PY{n+nb}{print}\PY{p}{(}\PY{l+s+s1}{\PYZsq{}}\PY{l+s+s1}{Амплітуда зовнішньої сили: f0 = }\PY{l+s+si}{\PYZob{}\PYZcb{}}\PY{l+s+s1}{\PYZsq{}}\PY{o}{.}\PY{n}{format}\PY{p}{(}\PY{n}{f0}\PY{p}{)}\PY{p}{)}
             \PY{n+nb}{print}\PY{p}{(}\PY{l+s+s1}{\PYZsq{}}\PY{l+s+s1}{Початкове положення: x0 = }\PY{l+s+si}{\PYZob{}\PYZcb{}}\PY{l+s+s1}{\PYZsq{}}\PY{o}{.}\PY{n}{format}\PY{p}{(}\PY{n}{x0}\PY{p}{)}\PY{p}{)}
             \PY{n+nb}{print}\PY{p}{(}\PY{l+s+s2}{\PYZdq{}}\PY{l+s+s2}{Початкова швидкіть: x0}\PY{l+s+s2}{\PYZsq{}}\PY{l+s+s2}{ = }\PY{l+s+si}{\PYZob{}\PYZcb{}}\PY{l+s+s2}{\PYZdq{}}\PY{o}{.}\PY{n}{format}\PY{p}{(}\PY{n}{x00}\PY{p}{)}\PY{p}{)}
             
             \PY{c+c1}{\PYZsh{} plot results}
             \PY{n}{fig}\PY{p}{,} \PY{n}{ax} \PY{o}{=} \PY{n}{plt}\PY{o}{.}\PY{n}{subplots}\PY{p}{(}\PY{n}{nrows}\PY{o}{=}\PY{l+m+mi}{1}\PY{p}{,} \PY{n}{ncols}\PY{o}{=}\PY{l+m+mi}{2}\PY{p}{)}
             \PY{n}{ax}\PY{p}{[}\PY{l+m+mi}{0}\PY{p}{]}\PY{o}{.}\PY{n}{plot}\PY{p}{(}\PY{n}{t}\PY{p}{,} \PY{n}{x}\PY{p}{[}\PY{p}{:}\PY{p}{,}\PY{l+m+mi}{0}\PY{p}{]}\PY{p}{)}
             \PY{n}{ax}\PY{p}{[}\PY{l+m+mi}{0}\PY{p}{]}\PY{o}{.}\PY{n}{set\PYZus{}xlabel}\PY{p}{(}\PY{l+s+s1}{\PYZsq{}}\PY{l+s+s1}{t}\PY{l+s+s1}{\PYZsq{}}\PY{p}{)}
             \PY{n}{ax}\PY{p}{[}\PY{l+m+mi}{0}\PY{p}{]}\PY{o}{.}\PY{n}{set\PYZus{}ylabel}\PY{p}{(}\PY{l+s+s1}{\PYZsq{}}\PY{l+s+s1}{x(t)}\PY{l+s+s1}{\PYZsq{}}\PY{p}{)}
             \PY{n}{ax}\PY{p}{[}\PY{l+m+mi}{1}\PY{p}{]}\PY{o}{.}\PY{n}{plot}\PY{p}{(}\PY{n}{x}\PY{p}{[}\PY{p}{:}\PY{p}{,}\PY{l+m+mi}{0}\PY{p}{]}\PY{p}{,} \PY{n}{x}\PY{p}{[}\PY{p}{:}\PY{p}{,}\PY{l+m+mi}{1}\PY{p}{]}\PY{p}{)}
             \PY{n}{ax}\PY{p}{[}\PY{l+m+mi}{1}\PY{p}{]}\PY{o}{.}\PY{n}{set\PYZus{}xlabel}\PY{p}{(}\PY{l+s+s1}{\PYZsq{}}\PY{l+s+s1}{x(t)}\PY{l+s+s1}{\PYZsq{}}\PY{p}{)}
             \PY{n}{ax}\PY{p}{[}\PY{l+m+mi}{1}\PY{p}{]}\PY{o}{.}\PY{n}{set\PYZus{}ylabel}\PY{p}{(}\PY{l+s+s2}{\PYZdq{}}\PY{l+s+s2}{x}\PY{l+s+s2}{\PYZsq{}}\PY{l+s+s2}{(t)}\PY{l+s+s2}{\PYZdq{}}\PY{p}{)}
             \PY{n}{plt}\PY{o}{.}\PY{n}{show}\PY{p}{(}\PY{p}{)}
         
         \PY{k}{def} \PY{n+nf}{plot\PYZus{}model3}\PY{p}{(}\PY{n}{delta}\PY{p}{,} \PY{n}{omega0}\PY{p}{,} \PY{n}{omega}\PY{p}{,} \PY{n}{f0}\PY{p}{,} \PY{n}{x0}\PY{p}{,} \PY{n}{x00}\PY{p}{)}\PY{p}{:}
             \PY{n}{description3} \PY{o}{=} \PY{n}{Label}\PY{p}{(}\PY{n}{value}\PY{o}{=}\PY{l+s+s2}{\PYZdq{}}\PY{l+s+s2}{\PYZdl{}x}\PY{l+s+s2}{\PYZsq{}}\PY{l+s+s2}{\PYZsq{}}\PY{l+s+s2}{+2δ x}\PY{l+s+s2}{\PYZsq{}}\PY{l+s+s2}{+ω\PYZus{}0\PYZca{}2 x=f\PYZus{}0 cos(ωt)\PYZdl{}}\PY{l+s+s2}{\PYZdq{}}\PY{p}{)}
             \PY{n}{δ\PYZus{}val} \PY{o}{=} \PY{n}{FloatText}\PY{p}{(}\PY{n}{value}\PY{o}{=}\PY{n}{delta}\PY{p}{,} \PY{n}{description} \PY{o}{=} \PYZbs{}
             \PY{l+s+s1}{\PYZsq{}}\PY{l+s+s1}{Коефіцієнт згасання: \PYZdl{}δ = \PYZdl{}}\PY{l+s+s1}{\PYZsq{}}\PY{p}{,} \PY{n}{step}\PY{o}{=}\PY{l+m+mf}{0.1}\PY{p}{,} \PY{n}{style} \PY{o}{=} \PY{n}{style}\PY{p}{,} \PY{n}{layout} \PY{o}{=} \PY{n}{layout}\PY{p}{)}
             \PY{n}{ω0\PYZus{}val} \PY{o}{=} \PY{n}{FloatText}\PY{p}{(}\PY{n}{value}\PY{o}{=}\PY{n}{omega0}\PY{p}{,} \PY{n}{description} \PY{o}{=} \PYZbs{}
             \PY{l+s+s1}{\PYZsq{}}\PY{l+s+s1}{Власна частота: \PYZdl{}ω\PYZus{}0 = \PYZdl{}}\PY{l+s+s1}{\PYZsq{}}\PY{p}{,} \PY{n}{step}\PY{o}{=}\PY{l+m+mf}{0.1}\PY{p}{,} \PY{n}{style} \PY{o}{=} \PY{n}{style}\PY{p}{,} \PY{n}{layout} \PY{o}{=} \PY{n}{layout}\PY{p}{)}
             \PY{n}{ω\PYZus{}val} \PY{o}{=} \PY{n}{FloatText}\PY{p}{(}\PY{n}{value}\PY{o}{=}\PY{n}{omega}\PY{p}{,} \PY{n}{description} \PY{o}{=} \PYZbs{}
             \PY{l+s+s1}{\PYZsq{}}\PY{l+s+s1}{Частота зовнішньої сили: \PYZdl{}ω = \PYZdl{}}\PY{l+s+s1}{\PYZsq{}}\PY{p}{,} \PY{n}{step}\PY{o}{=}\PY{l+m+mf}{0.1}\PY{p}{,}
                               \PY{n}{style} \PY{o}{=} \PY{n}{style}\PY{p}{,} \PY{n}{layout} \PY{o}{=} \PY{n}{layout}\PY{p}{)}
             \PY{n}{f0\PYZus{}val} \PY{o}{=} \PY{n}{FloatText}\PY{p}{(}\PY{n}{value}\PY{o}{=}\PY{n}{f0}\PY{p}{,} \PY{n}{description} \PY{o}{=} \PYZbs{}
             \PY{l+s+s1}{\PYZsq{}}\PY{l+s+s1}{Амплітуда зовнішньої сили: \PYZdl{}f\PYZus{}0 = \PYZdl{}}\PY{l+s+s1}{\PYZsq{}}\PY{p}{,} \PY{n}{step}\PY{o}{=}\PY{l+m+mf}{0.1}\PY{p}{,}
                                \PY{n}{style} \PY{o}{=} \PY{n}{style}\PY{p}{,} \PY{n}{layout} \PY{o}{=} \PY{n}{layout}\PY{p}{)}
             \PY{n}{x0\PYZus{}val} \PY{o}{=} \PY{n}{FloatText}\PY{p}{(}\PY{n}{value}\PY{o}{=}\PY{n}{x0}\PY{p}{,} \PY{n}{description} \PY{o}{=} \PYZbs{}
             \PY{l+s+s1}{\PYZsq{}}\PY{l+s+s1}{Початкове положення: \PYZdl{}x\PYZus{}0 = \PYZdl{}}\PY{l+s+s1}{\PYZsq{}}\PY{p}{,} \PY{n}{step}\PY{o}{=}\PY{l+m+mf}{0.1}\PY{p}{,}
                                \PY{n}{style} \PY{o}{=} \PY{n}{style}\PY{p}{,} \PY{n}{layout} \PY{o}{=} \PY{n}{layout}\PY{p}{)}
             \PY{n}{x00\PYZus{}val} \PY{o}{=} \PY{n}{FloatText}\PY{p}{(}\PY{n}{value}\PY{o}{=}\PY{n}{x00}\PY{p}{,} \PY{n}{description} \PY{o}{=} \PYZbs{}
             \PY{l+s+s2}{\PYZdq{}}\PY{l+s+s2}{Початкова швидкіть: \PYZdl{}x\PYZus{}0}\PY{l+s+s2}{\PYZsq{}}\PY{l+s+s2}{ = \PYZdl{}}\PY{l+s+s2}{\PYZdq{}}\PY{p}{,} \PY{n}{step}\PY{o}{=}\PY{l+m+mf}{0.1}\PY{p}{,}
                                 \PY{n}{style} \PY{o}{=} \PY{n}{style}\PY{p}{,} \PY{n}{layout} \PY{o}{=} \PY{n}{layout}\PY{p}{)}
             \PY{n}{models}\PY{p}{[}\PY{l+m+mi}{2}\PY{p}{]} \PY{o}{=} \PY{n}{VBox}\PY{p}{(}\PY{p}{[}\PY{n}{description3}\PY{p}{,} \PY{n}{interactive}\PY{p}{(}\PY{n}{model3}\PY{p}{,} \PY{n}{δ}\PY{o}{=}\PY{n}{δ\PYZus{}val}\PY{p}{,} \PY{n}{ω0}\PY{o}{=}\PY{n}{ω0\PYZus{}val}\PY{p}{,}
                                                         \PY{n}{ω}\PY{o}{=}\PY{n}{ω\PYZus{}val}\PY{p}{,} \PY{n}{f0}\PY{o}{=}\PY{n}{f0\PYZus{}val}\PY{p}{,}
                                                         \PY{n}{x0}\PY{o}{=}\PY{n}{x0\PYZus{}val}\PY{p}{,} \PY{n}{x00}\PY{o}{=}\PY{n}{x00\PYZus{}val}\PY{p}{,}
                                                         \PY{n}{continuous\PYZus{}update}\PY{o}{=}\PY{k+kc}{True}\PY{p}{)}\PY{p}{]}\PY{p}{)}
             \PY{n}{display}\PY{p}{(}\PY{n}{models}\PY{p}{[}\PY{l+m+mi}{2}\PY{p}{]}\PY{p}{)}
             
             
         \PY{k}{def} \PY{n+nf}{model4}\PY{p}{(}\PY{n}{αx}\PY{p}{,} \PY{n}{αy}\PY{p}{,} \PY{n}{βx}\PY{p}{,} \PY{n}{βy}\PY{p}{,} \PY{n}{x0}\PY{p}{,} \PY{n}{y0}\PY{p}{)}\PY{p}{:}
             \PY{k}{def} \PY{n+nf}{model4\PYZus{}eq}\PY{p}{(}\PY{n}{x}\PY{p}{,}\PY{n}{t}\PY{p}{)}\PY{p}{:}
                 \PY{k}{return} \PY{p}{[}\PY{p}{(}\PY{n}{αx} \PY{o}{*} \PY{n}{x}\PY{p}{[}\PY{l+m+mi}{1}\PY{p}{]} \PY{o}{\PYZhy{}} \PY{n}{βx}\PY{p}{)} \PY{o}{*} \PY{n}{x}\PY{p}{[}\PY{l+m+mi}{0}\PY{p}{]}\PY{p}{,} \PY{p}{(}\PY{n}{αy} \PY{o}{\PYZhy{}} \PY{n}{x}\PY{p}{[}\PY{l+m+mi}{0}\PY{p}{]} \PY{o}{*} \PY{n}{βy}\PY{p}{)} \PY{o}{*} \PY{n}{x}\PY{p}{[}\PY{l+m+mi}{1}\PY{p}{]}\PY{p}{]}
             
             \PY{n}{t} \PY{o}{=} \PY{n}{np}\PY{o}{.}\PY{n}{linspace}\PY{p}{(}\PY{l+m+mi}{0}\PY{p}{,} \PY{l+m+mi}{50}\PY{p}{,} \PY{n}{num}\PY{o}{=}\PY{l+m+mi}{1000}\PY{p}{)}
         
             \PY{c+c1}{\PYZsh{} solve ODE}
             \PY{n}{x} \PY{o}{=} \PY{n}{odeint}\PY{p}{(}\PY{n}{model4\PYZus{}eq}\PY{p}{,} \PY{n}{np}\PY{o}{.}\PY{n}{array}\PY{p}{(}\PY{p}{[}\PY{n}{x0}\PY{p}{,} \PY{n}{y0}\PY{p}{]}\PY{p}{)}\PY{p}{,} \PY{n}{t}\PY{p}{)}
             
             \PY{n+nb}{print}\PY{p}{(}\PY{l+s+s1}{\PYZsq{}}\PY{l+s+s1}{\PYZdq{}}\PY{l+s+s1}{Норма споживання}\PY{l+s+s1}{\PYZdq{}}\PY{l+s+s1}{ жертв: αx = }\PY{l+s+si}{\PYZob{}\PYZcb{}}\PY{l+s+s1}{\PYZsq{}}\PY{o}{.}\PY{n}{format}\PY{p}{(}\PY{n}{αx}\PY{p}{)}\PY{p}{)}
             \PY{n+nb}{print}\PY{p}{(}\PY{l+s+s1}{\PYZsq{}}\PY{l+s+s1}{Природна народжуваність жертв: αy = }\PY{l+s+si}{\PYZob{}\PYZcb{}}\PY{l+s+s1}{\PYZsq{}}\PY{o}{.}\PY{n}{format}\PY{p}{(}\PY{n}{αy}\PY{p}{)}\PY{p}{)} 
             \PY{n+nb}{print}\PY{p}{(}\PY{l+s+s1}{\PYZsq{}}\PY{l+s+s1}{Природна смертність хижаків: βx = }\PY{l+s+si}{\PYZob{}\PYZcb{}}\PY{l+s+s1}{\PYZsq{}}\PY{o}{.}\PY{n}{format}\PY{p}{(}\PY{n}{βx}\PY{p}{)}\PY{p}{)} 
             \PY{n+nb}{print}\PY{p}{(}\PY{l+s+s1}{\PYZsq{}}\PY{l+s+s1}{\PYZdq{}}\PY{l+s+s1}{Норма споживаності}\PY{l+s+s1}{\PYZdq{}}\PY{l+s+s1}{ жертв: βy = }\PY{l+s+si}{\PYZob{}\PYZcb{}}\PY{l+s+s1}{\PYZsq{}}\PY{o}{.}\PY{n}{format}\PY{p}{(}\PY{n}{βy}\PY{p}{)}\PY{p}{)} 
             \PY{n+nb}{print}\PY{p}{(}\PY{l+s+s1}{\PYZsq{}}\PY{l+s+s1}{Початкова кількість хижаків: x0 = }\PY{l+s+si}{\PYZob{}\PYZcb{}}\PY{l+s+s1}{\PYZsq{}}\PY{o}{.}\PY{n}{format}\PY{p}{(}\PY{n}{x0}\PY{p}{)}\PY{p}{)} 
             \PY{n+nb}{print}\PY{p}{(}\PY{l+s+s2}{\PYZdq{}}\PY{l+s+s2}{Початкова кількість жертв: y0 = }\PY{l+s+si}{\PYZob{}\PYZcb{}}\PY{l+s+s2}{\PYZdq{}}\PY{o}{.}\PY{n}{format}\PY{p}{(}\PY{n}{y0}\PY{p}{)}\PY{p}{)}
             
             \PY{c+c1}{\PYZsh{} plot results}
             \PY{n}{fig}\PY{p}{,} \PY{n}{ax} \PY{o}{=} \PY{n}{plt}\PY{o}{.}\PY{n}{subplots}\PY{p}{(}\PY{n}{nrows}\PY{o}{=}\PY{l+m+mi}{1}\PY{p}{,} \PY{n}{ncols}\PY{o}{=}\PY{l+m+mi}{2}\PY{p}{)}
             \PY{n}{ax}\PY{p}{[}\PY{l+m+mi}{0}\PY{p}{]}\PY{o}{.}\PY{n}{plot}\PY{p}{(}\PY{n}{t}\PY{p}{,} \PY{n}{x}\PY{p}{[}\PY{p}{:}\PY{p}{,}\PY{l+m+mi}{0}\PY{p}{]}\PY{p}{,} \PY{n}{label} \PY{o}{=} \PY{l+s+s1}{\PYZsq{}}\PY{l+s+s1}{x(t)}\PY{l+s+s1}{\PYZsq{}}\PY{p}{)}
             \PY{n}{ax}\PY{p}{[}\PY{l+m+mi}{0}\PY{p}{]}\PY{o}{.}\PY{n}{plot}\PY{p}{(}\PY{n}{t}\PY{p}{,} \PY{n}{x}\PY{p}{[}\PY{p}{:}\PY{p}{,}\PY{l+m+mi}{1}\PY{p}{]}\PY{p}{,} \PY{n}{label} \PY{o}{=} \PY{l+s+s1}{\PYZsq{}}\PY{l+s+s1}{y(t)}\PY{l+s+s1}{\PYZsq{}}\PY{p}{)}
             \PY{n}{ax}\PY{p}{[}\PY{l+m+mi}{0}\PY{p}{]}\PY{o}{.}\PY{n}{legend}\PY{p}{(}\PY{p}{)}
             \PY{n}{ax}\PY{p}{[}\PY{l+m+mi}{0}\PY{p}{]}\PY{o}{.}\PY{n}{set\PYZus{}xlabel}\PY{p}{(}\PY{l+s+s1}{\PYZsq{}}\PY{l+s+s1}{t}\PY{l+s+s1}{\PYZsq{}}\PY{p}{)}
             \PY{n}{ax}\PY{p}{[}\PY{l+m+mi}{1}\PY{p}{]}\PY{o}{.}\PY{n}{plot}\PY{p}{(}\PY{n}{x}\PY{p}{[}\PY{p}{:}\PY{p}{,}\PY{l+m+mi}{0}\PY{p}{]}\PY{p}{,} \PY{n}{x}\PY{p}{[}\PY{p}{:}\PY{p}{,}\PY{l+m+mi}{1}\PY{p}{]}\PY{p}{)}
             \PY{k}{if} \PY{n}{βy} \PY{o}{!=} \PY{l+m+mi}{0} \PY{o+ow}{and} \PY{n}{αx} \PY{o}{!=} \PY{l+m+mi}{0}\PY{p}{:}
                 \PY{n}{ax}\PY{p}{[}\PY{l+m+mi}{1}\PY{p}{]}\PY{o}{.}\PY{n}{scatter}\PY{p}{(}\PY{n}{αy}\PY{o}{/}\PY{n}{βy}\PY{p}{,} \PY{n}{βx}\PY{o}{/}\PY{n}{αx}\PY{p}{,} \PY{n}{s}\PY{o}{=}\PY{l+m+mi}{25}\PY{p}{,} \PY{n}{c}\PY{o}{=}\PY{l+m+mi}{4}\PY{p}{)}
             \PY{n}{ax}\PY{p}{[}\PY{l+m+mi}{1}\PY{p}{]}\PY{o}{.}\PY{n}{set\PYZus{}xlabel}\PY{p}{(}\PY{l+s+s1}{\PYZsq{}}\PY{l+s+s1}{x(t)}\PY{l+s+s1}{\PYZsq{}}\PY{p}{)}
             \PY{n}{ax}\PY{p}{[}\PY{l+m+mi}{1}\PY{p}{]}\PY{o}{.}\PY{n}{set\PYZus{}ylabel}\PY{p}{(}\PY{l+s+s2}{\PYZdq{}}\PY{l+s+s2}{y(t)}\PY{l+s+s2}{\PYZdq{}}\PY{p}{)}
             \PY{n}{plt}\PY{o}{.}\PY{n}{show}\PY{p}{(}\PY{p}{)}
         
         \PY{k}{def} \PY{n+nf}{plot\PYZus{}model4}\PY{p}{(}\PY{n}{alpha\PYZus{}x}\PY{p}{,} \PY{n}{alpha\PYZus{}y}\PY{p}{,} \PY{n}{beta\PYZus{}x}\PY{p}{,} \PY{n}{beta\PYZus{}y}\PY{p}{,} \PY{n}{x0}\PY{p}{,} \PY{n}{y0}\PY{p}{)}\PY{p}{:}
             \PY{n}{description4} \PY{o}{=} \PY{n}{Label}\PY{p}{(}\PY{n}{value}\PY{o}{=}\PYZbs{}
             \PY{l+s+sa}{r}\PY{l+s+sd}{\PYZdq{}\PYZdq{}\PYZdq{}\PYZdl{} \PYZbs{}bigg\PYZbs{}\PYZob{} \PYZbs{}matrix\PYZob{}x\PYZsq{}=(α\PYZus{}x y \PYZhy{} β\PYZus{}x)x \PYZbs{}cr y\PYZsq{}=(α\PYZus{}y \PYZhy{} β\PYZus{}y x)y\PYZcb{} \PYZdl{}\PYZdq{}\PYZdq{}\PYZdq{}}\PY{p}{,}
                                  \PY{n}{layout} \PY{o}{=} \PY{n}{Layout}\PY{p}{(}\PY{n}{height} \PY{o}{=} \PY{l+s+s1}{\PYZsq{}}\PY{l+s+s1}{50px}\PY{l+s+s1}{\PYZsq{}}\PY{p}{)}\PY{p}{)}
             \PY{n}{αx\PYZus{}val} \PY{o}{=} \PY{n}{FloatText}\PY{p}{(}\PY{n}{value}\PY{o}{=}\PY{n}{alpha\PYZus{}x}\PY{p}{,} \PY{n}{description} \PY{o}{=} \PYZbs{}
            \PY{l+s+s1}{\PYZsq{}}\PY{l+s+s1}{\PYZdq{}}\PY{l+s+s1}{Норма споживання}\PY{l+s+s1}{\PYZdq{}}\PY{l+s+s1}{ жертв: \PYZdl{}α\PYZus{}x = \PYZdl{}}\PY{l+s+s1}{\PYZsq{}}\PY{p}{,} \PY{n}{step}\PY{o}{=}\PY{l+m+mf}{0.1}\PY{p}{,} \PY{n}{style} \PY{o}{=} \PY{n}{style}\PY{p}{,} 
                                \PY{n}{layout} \PY{o}{=} \PY{n}{layout}\PY{p}{)}
             \PY{n}{αy\PYZus{}val} \PY{o}{=} \PY{n}{FloatText}\PY{p}{(}\PY{n}{value}\PY{o}{=}\PY{n}{alpha\PYZus{}y}\PY{p}{,} \PY{n}{description} \PY{o}{=} \PYZbs{}
            \PY{l+s+s1}{\PYZsq{}}\PY{l+s+s1}{Природна народжуваність жертв: \PYZdl{}α\PYZus{}y = \PYZdl{}}\PY{l+s+s1}{\PYZsq{}}\PY{p}{,} \PY{n}{step}\PY{o}{=}\PY{l+m+mf}{0.1}\PY{p}{,} \PY{n}{style} \PY{o}{=} \PY{n}{style}\PY{p}{,} 
                                \PY{n}{layout} \PY{o}{=} \PY{n}{layout}\PY{p}{)}
             \PY{n}{βx\PYZus{}val} \PY{o}{=} \PY{n}{FloatText}\PY{p}{(}\PY{n}{value}\PY{o}{=}\PY{n}{beta\PYZus{}x}\PY{p}{,} \PY{n}{description} \PY{o}{=} \PYZbs{}
            \PY{l+s+s1}{\PYZsq{}}\PY{l+s+s1}{Природна смертність хижаків: \PYZdl{}β\PYZus{}x = \PYZdl{}}\PY{l+s+s1}{\PYZsq{}}\PY{p}{,} \PY{n}{step}\PY{o}{=}\PY{l+m+mf}{0.1}\PY{p}{,} \PY{n}{style} \PY{o}{=} \PY{n}{style}\PY{p}{,} 
                                \PY{n}{layout} \PY{o}{=} \PY{n}{layout}\PY{p}{)}
             \PY{n}{βy\PYZus{}val} \PY{o}{=} \PY{n}{FloatText}\PY{p}{(}\PY{n}{value}\PY{o}{=}\PY{n}{beta\PYZus{}y}\PY{p}{,} \PY{n}{description} \PY{o}{=} \PYZbs{}
            \PY{l+s+s1}{\PYZsq{}}\PY{l+s+s1}{\PYZdq{}}\PY{l+s+s1}{Норма споживаності}\PY{l+s+s1}{\PYZdq{}}\PY{l+s+s1}{ жертв: \PYZdl{}β\PYZus{}y = \PYZdl{}}\PY{l+s+s1}{\PYZsq{}}\PY{p}{,} \PY{n}{step}\PY{o}{=}\PY{l+m+mf}{0.1}\PY{p}{,} \PY{n}{style} \PY{o}{=} \PY{n}{style}\PY{p}{,} 
                                \PY{n}{layout} \PY{o}{=} \PY{n}{layout}\PY{p}{)}
             \PY{n}{x0\PYZus{}val} \PY{o}{=} \PY{n}{FloatText}\PY{p}{(}\PY{n}{value}\PY{o}{=}\PY{n}{x0}\PY{p}{,} \PY{n}{description} \PY{o}{=} \PYZbs{}
            \PY{l+s+s1}{\PYZsq{}}\PY{l+s+s1}{Початкова кількість хижаків: \PYZdl{}x\PYZus{}0 = \PYZdl{}}\PY{l+s+s1}{\PYZsq{}}\PY{p}{,} \PY{n}{step}\PY{o}{=}\PY{l+m+mf}{0.1}\PY{p}{,} \PY{n}{style} \PY{o}{=} \PY{n}{style}\PY{p}{,} 
                                \PY{n}{layout} \PY{o}{=} \PY{n}{layout}\PY{p}{)}
             \PY{n}{y0\PYZus{}val} \PY{o}{=} \PY{n}{FloatText}\PY{p}{(}\PY{n}{value}\PY{o}{=}\PY{n}{y0}\PY{p}{,} \PY{n}{description} \PY{o}{=} \PYZbs{}
            \PY{l+s+s2}{\PYZdq{}}\PY{l+s+s2}{Початкова кількість жертв: \PYZdl{}y\PYZus{}0 = \PYZdl{}}\PY{l+s+s2}{\PYZdq{}}\PY{p}{,} \PY{n}{step}\PY{o}{=}\PY{l+m+mf}{0.1}\PY{p}{,} \PY{n}{style} \PY{o}{=} \PY{n}{style}\PY{p}{,}
                                \PY{n}{layout} \PY{o}{=} \PY{n}{layout}\PY{p}{)}
             \PY{n}{models}\PY{p}{[}\PY{l+m+mi}{3}\PY{p}{]} \PY{o}{=} \PY{n}{VBox}\PY{p}{(}\PY{p}{[}\PY{n}{description4}\PY{p}{,} \PY{n}{interactive}\PY{p}{(}\PY{n}{model4}\PY{p}{,} \PY{n}{αx}\PY{o}{=}\PY{n}{αx\PYZus{}val}\PY{p}{,} \PY{n}{αy}\PY{o}{=}\PY{n}{αy\PYZus{}val}\PY{p}{,}
                                                         \PY{n}{βx}\PY{o}{=}\PY{n}{βx\PYZus{}val}\PY{p}{,} \PY{n}{βy}\PY{o}{=}\PY{n}{βy\PYZus{}val}\PY{p}{,}
                                                         \PY{n}{x0}\PY{o}{=}\PY{n}{x0\PYZus{}val}\PY{p}{,} \PY{n}{y0}\PY{o}{=}\PY{n}{y0\PYZus{}val}\PY{p}{,}
                                                         \PY{n}{continuous\PYZus{}update}\PY{o}{=}\PY{k+kc}{True}\PY{p}{)}\PY{p}{]}\PY{p}{)}
             \PY{n}{display}\PY{p}{(}\PY{n}{models}\PY{p}{[}\PY{l+m+mi}{3}\PY{p}{]}\PY{p}{)}
\end{Verbatim}



    % Add a bibliography block to the postdoc
    
    
    
    \end{document}
